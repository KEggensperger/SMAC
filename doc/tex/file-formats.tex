\documentclass[manual.tex]{subfiles} 
\begin{document}

%%%%%%%%%%%%%%%%%%%%%%%%%%%%%%%%%%%%%%%%%%%%%%%%%%%%%%%%%%%%%%%%%%%%
\subsection{Option Files}
%%%%%%%%%%%%%%%%%%%%%%%%%%%%%%%%%%%%%%%%%%%%%%%%%%%%%%%%%%%%%%%%%%%%
Option Files are a way of saving a different set of values 
frequently used with SMAC without having to specify them on every execution. 
The general format for an option file is the name
of the configuration option (without the two dashes), an equal sign, and then the value (for booleans it should be true or false, lowercase). Currently options that take multiple arguments are not supported. Additionally you can not use aliases that are single dashed (\eg{ to override the Experiment Directory, you must use \textbf{-~$\!\!$-experiment-dir} and not \textbf{-e}})  

When using Option Files it is important that no two files (including the Scenario File), specify the same option, the resulting configuration is undefined, and in general this will not throw an error.

%%%%%%%%%%%%%%%%%%%%%%%%%%%%%%%%%%%%%%%%%%%%%%%%%%%%%%%%%%%%%%%%%%%%
\subsubsection{Scenario File}
%%%%%%%%%%%%%%%%%%%%%%%%%%%%%%%%%%%%%%%%%%%%%%%%%%%%%%%%%%%%%%%%%%%%

The Scenario Option File, or Scenario File, is the recommended way of configuring SMAC \footnote{Nothing in general prevents you from specifying non-scenario options in these files, but in general you should restrict your files to these.}.
The Scenario Files used in SMAC are backwards compatible with ParamILS and the name of option names here reflect that\footnote{Every option name listed here is in fact an alias for an existing option listed in the section \ref{sec:options-ref} and it is entirely possible to use SMAC without using Scenario Files.}. \textsc{Note:} \textbf{cutoff\_length} is not currently supported.

\begin{description}
\item [algo] An algorithm executable or wrapper script around
an algorithm that conforms with the input/output format specified
in section \ref{sec:exec-spec}. The string here should be invokable via the system shell.
\item [{execdir}] Directory to execute \texttt{<algo>} from: (\ie{ ``cd
\texttt{<execdir>}; \texttt{<algo>}'' })
\item [{deterministic}] A boolean that governs whether or not the algorithm should be treated as deterministic. For backwards compatibility with ParamILS, this option also supports using 0 for false, and 1 for true. SMAC will never invoke the target algorithm more than once for any given instance, seed and configuration. If this is set to \texttt{true}, SMAC will never invoke the target algorithm more than once for any given instance and configuration.

\item [{run\_obj}] Determines how to convert the resulting output line
into a scalar quantifying how ``good'' a single algorithm execution
is, (\eg{ how long it took to execute, how good of a solution it found,
etc...}). SMAC will attempt to \textit{minimize} this objective. Currently implemented objectives are the following:
\item [{
\begin{tabular}{|c|c|}
\hline 
Name & Description\tabularnewline
\hline 
\hline 
RUNTIME & The reported runtime of the algorithm.\tabularnewline
\hline 
QUALITY & The reported quality of the algorithm.\tabularnewline
\hline 
\end{tabular}}]~
\item [{overall\_obj}] While \textbf{run\_obj} defines the objective function
for a single algorithm run, \textbf{overall\_obj} defines how those
single objectives are combined to reach a single scalar value to compare
two parameter configurations. Implemented examples for this are as
follows:
\item [{%
\begin{tabular}{|c|c|}
\hline 
Name & Description\tabularnewline
\hline 
\hline 
MEAN & The mean of the values\tabularnewline
\hline
MEAN1000 & Unsuccessful runs are counted as 1000 $\times$ \textbf{cutoff\_time}\tabularnewline
\hline 
MEAN10 & Unsuccessful runs are counted as 10 $\times$ \textbf{cutoff\_time}\tabularnewline
\hline 
\hline 
\end{tabular}}]~
\item [{cutoff\_time}] The CPU time after which a single algorithm execution
will be terminated as unsuccess (and treated as a \textbf{TIMEOUT}). This is an important parameter: If chosen too high, lots of time will be wasted with unsuccessful
runs. If chosen too low the optimization is biased to perform well
on easy instances only. 
%\item [{cutoff\_length}] The runlength after which a single algorithm execution
%will be terminated unsuccesfully. The actual semantic meaning of this
%value is up to the target algorithm. NOT SUPPORTED IN THIS VERSION)
\item [{tunerTimeout}] The limit of the CPU time allowed for configuration (\ie{The sum of all algorithm runtimes, and by default the sum of the CPU time of SMAC itself}).
\item [{paramfile}] Specifies the file with the parameters of the algorithm.
The format of this file is covered in Section \ref{sec:paramfile}.
\item [{outdir}] Specifies the directory SMAC should write its results
to. 
\item [{instance\_file}] Specifies the file containing the list of problem instances (and possibly seeds) for SMAC to use during the \emph{Automatic Configuration Phase}. The ParamILS parameter \textbf{instance\_seed\_file} aliases this one and the format is auto-detected. The format of these files is covered in section \ref{sec:instance_file_format}.
\item [{test\_instance\_file}] Specifies the file containing the list of problem instances (and possibly seeds) for SMAC to use during \emph{Validation Phase}. The ParamILS parameter \textbf{test\_instance\_seed\_file} aliases this one and the format is auto-detected. The format of these files is covered in section \ref{sec:instance_file_format}.
\item [{feature\_file}] Specifies the a file with the features for the instances in the \textbf{instance\_file} and possibly the \textbf{test\_instance\_file} \footnote{The Validator will load features into memory for test instances if they exist.}. The format of this file is covered in section \ref{sec:feature_file_format}.

\end{description}

%%%%%%%%%%%%%%%%%%%%%%%%%%%%%%%%%%%%%%%%%%%%%%%%%%%%%%%%%%%%%%%%%%%%
\subsection{Instance File Format} \label{sec:instance_file_format}
%%%%%%%%%%%%%%%%%%%%%%%%%%%%%%%%%%%%%%%%%%%%%%%%%%%%%%%%%%%%%%%%%%%%


The files used by the \textbf{instance\_file} \& \textbf{test\_instance\_file} options 
come in four potential formats, all of which are CSV based\footnote{Specifically each cell should be double-quoted (\ie{''}), and use a comma as a cell delimiter. SMAC also supports the old method of reading files that use space as a cell delimiter and do not enclose values. However these files cannot handle \textbf{Instance Name}'s that contain spaces.}. Before specifying the formats it is important to note the three kinds of information that are specified with instances \footnote{Features which are required for SMAC but not ParamILS are specified in a seperate file see section \ref{sec:feature_file_format}.}.

\begin{description}
\item[Instance Name] The name of the instance that was selected. This should be meaningful to the target algorithm we are configuring \footnote{Generally \textbf{Instance Names} reference specific files on disk.}.
\item[Instance Specific Information] A free form text string (with no spaces or line breaks) that will be passed to the Target Algorithm whenever executed.
\item[Seed] A specific seed to use when executing the target algorithm.
\end{description}

The possible formats are as follows, and depend on what information you'd like to specify.

\begin{enumerate}
\item	Each line specifies only a unique \textbf{Instance Name}. No \textbf{Instance Specific Information} will be used, and \textbf{Seed}'s will be automatically generated.

\item  Each line specifies a \textbf{Seed} followed by the \textbf{Instance Name}. Every line must be unique, but for each \textbf{Instance Name} additional seeds will be used in order, when that instance is selected.

\item Each line specifies a \textbf{Instance Name} followed by the \textbf{Instance Specific Information}. Every \textbf{Instance Name} must be unique, \textbf{Seed}'s will be automatically generated.

\item Each line specifies a \textbf{Seed} followed by the \textbf{Instance Name} followed by the \textbf{Instance Specific Information}. Every line must be unique, and furthermore, for all \textbf{Instance Name}'s the \textbf{Instance Specific Information} must be the same for all \textbf{Seed} values (\ie{You cannot specify different instance specific information that is a function of the seed used}).

\end{enumerate}

%%%%%%%%%%%%%%%%%%%%%%%%%%%%%%%%%%%%%%%%%%%%%%%%%%%%%%%%%%%%%%%%%%%%
\subsection{Feature File Format}\label{sec:feature_file_format}
%%%%%%%%%%%%%%%%%%%%%%%%%%%%%%%%%%%%%%%%%%%%%%%%%%%%%%%%%%%%%%%%%%%%



The \textbf{feature\_file} specifies features that are to be used for instances. Feature Files are specified in CSV format, the first column of every row should list an \textbf{Instance Name} as it appears in the \textbf{instance\_file}. The subsequent columns should list \texttt{double} values specifying a computed continuous feature. By convention the value $-512$, and $-1024$ are used to signify that a feature value is missing or not applicable. All instances must have the same number of features.

At the top of the file there \textsc{must} appear a header row, the cell that appears above the instance names is unimportant, but for each feature a unique and \emph{non-numeric} (\ie{ it must contain atleast one letter}) feature name must be specified.

%%%%%%%%%%%%%%%%%%%%%%%%%%%%%%%%%%%%%%%%%%%%%%%%%%%%%%%%%%%%%%%%%%%%
\subsection{Algorithm Parameter File} \label{sec:paramfile}
%%%%%%%%%%%%%%%%%%%%%%%%%%%%%%%%%%%%%%%%%%%%%%%%%%%%%%%%%%%%%%%%%%%%

 \subfile{pcs-subfile}


\end{document}

