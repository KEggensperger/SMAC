\documentclass[11pt,letterpaper,oneside]{article}
%%%%%%%%%%%%%%%%%%%%%%%%%%%%%%%%%%%%%%%%%%%%%%
% GENERAL PACKAGES
%%%%%%%%%%%%%%%%%%%%%%%%%%%%%%%%%%%%%%%%%%%%%%

%\usepackage{epsfig}
\usepackage{subfigure}
\usepackage{url}
\urlstyle{sf}	% typeset urls in sans-serif

%\usepackage{amsmath}
%\usepackage{amsfonts}
%\usepackage{color}
%\usepackage{framed}

%\usepackage{makeidx}  % allows for indexgeneration
\usepackage{bm}
\usepackage{graphicx}
\usepackage[boxed, algoruled, vlined, linesnumbered]{algorithm2e} % noresetcount
\usepackage{times}
\usepackage{dsfont}
\usepackage[longnamesfirst,numbers]{natbib} %FH: longnamesfirst does not seem to work :(
\usepackage{microtype}
%\usepackage[longnamesfirst]{natbib}
\usepackage{amsmath, amssymb, amsfonts}

%Next 7 lines: tell natbib not to put bibliography onto new page
\makeatletter
\renewcommand\bibsection%
{
  \section*{\refname
    \@mkboth{\MakeUppercase{\refname}}{\MakeUppercase{\refname}}}
}
\makeatother

%% Define a new style for URLs that will use a smaller font.
%% use a different one for footnotes (requires manual switching)
\makeatletter
\def\url@newstyle{%
  \@ifundefined{selectfont}{\def\UrlFont{\sf}}{\def\UrlFont{\small}}}
\def\url@newFNstyle{%
  \@ifundefined{selectfont}{\def\UrlFont{\sf}}{\def\UrlFont{\scriptsize}}}
\makeatother
\urlstyle{new}


\newcommand{\todo}[1]{\textbf{TODO: #1}}
\newcommand{\todocrc}[1]{\note{TODOCRC: #1}}
\newcommand{\cmnt}[1]{\textbf{COMMENT: #1}}
\newcommand{\fhcrc}[1]{\note{FH CRC comment: #1}}


\newcommand{\SingleInstSaps}{{\small{\texttt{\textsc{Single\-Inst\-Saps}}}}}
\newcommand{\SingleInstSpear}{{\small{\texttt{\textsc{Single\-Inst\-Spear}}}}}
\newcommand{\Broad}{{\small{\texttt{\textsc{Broad}}}}}
\renewcommand{\AlTitleFnt}[1]{{\bf{}#1}}

%%%%%%%%%%%%%%%%%%%%%%%%%%%%%%%%%%%%%%%%%%%%%%
% CONVENIENT COMMANDS for commenting text
%%%%%%%%%%%%%%%%%%%%%%%%%%%%%%%%%%%%%%%%%%%%%%
\newcommand{\hide}[1]{}
\newcommand{\fh}[1]{{\bf{}FH: #1}}
\newcommand{\hh}[1]{\note{HH says: #1}}
\renewcommand{\hh}[1]{}
\newcommand{\klb}[1]{{\bf{}KLB: #1}}



%Configurators
\newcommand{\algofont}[1]{{\footnotesize{\textsc{#1}}}}
\newcommand{\paramils}{\algofont{Param\-ILS}}
%\newcommand{\paramils}{\textsc{Param\-ILS}}
\newcommand{\focusedils}{\algofont{Focused\-ILS}}
\newcommand{\basicils}{\algofont{Basic\-ILS}}
\newcommand{\gga}{\algofont{GGA}}
\newcommand{\smac}{\algofont{SMAC}}
\newcommand{\roar}{\algofont{ROAR}}
\newcommand{\tbspo}{\algofont{TB-SPO}}
\newcommand{\spop}{{\algofont{SPO$^+$}}}
\newcommand{\randomstar}{\algofont{Ran\-dom$^*$}}
\newcommand{\randomsearchstar}{\randomstar}
\newcommand{\frace}{{\algofont{F-Race}}}

\newcommand{\salgofont}[1]{{\scriptsize{\textsc{#1}}}}
\newcommand{\sgga}{\salgofont{GGA}}
\newcommand{\ssmac}{\salgofont{SMAC}}
\newcommand{\sroar}{\salgofont{ROAR}}
\newcommand{\stbspo}{\salgofont{TB-SPO}}
\newcommand{\sspop}{{\salgofont{SPO$^+$}}}

%SAT solvers
\newcommand{\spear}{\algofont{SPEAR}}
\newcommand{\saps}{\algofont{SAPS}}
\newcommand{\satenstein}{\algofont{SATenstein}}

%MIP solvers
\newcommand{\lpsolve}{\algofont{lp\-solve}}
\newcommand{\gurobi}{\algofont{Gu\-ro\-bi}}
\newcommand{\cplex}{\algofont{CPLEX}}
\newcommand{\ibmcplex}{\algofont{IBM ILOG CPLEX}}

%Instance sets
\newcommand{\shoe}{\textsc{SHOE}}
%\newcommand{\MASS}{\textsc{MASS}}
\newcommand{\QCP}{QCP}
\newcommand{\SWGCP}{SWGCP}
\newcommand{\MASS}{MASS}
\newcommand{\MIK}{MIK}
\newcommand{\CLS}{CLS}
\newcommand{\MJA}{MJA}
\newcommand{\corlat}{\textsc{CORLAT}}
\newcommand{\regionsonehundred}{\textsc{Regions100}}
\newcommand{\regionstwohundred}{\textsc{Regions200}}
\newcommand{\regionsseventy}{\textsc{Regions70}}

\newcommand{\SPEARSWGCP}{{\footnotesize{\texttt{\textsc{Spear-SWGCP}}}}}
\newcommand{\SPEARQCP}{{\footnotesize{\texttt{\textsc{Spear-QCP}}}}}

\newcommand{\SAPSSWGCP}{{\footnotesize{\texttt{\textsc{Saps-SWGCP}}}}}
\newcommand{\SAPSSWGCPhomog}{{\footnotesize{\texttt{\textsc{Saps-SW-hom}}}}}

\newcommand{\cplexregionsonehundred}{{\footnotesize{\texttt{\textsc{CPLEX-Regions100}}}}}
\newcommand{\cplexmik}{{\footnotesize{\texttt{\textsc{CPLEX-MIK}}}}}
\newcommand{\SAPSQCP}{{\footnotesize{\texttt{\textsc{Saps-QCP}}}}}
\newcommand{\SAPSQCPeasy}{{\footnotesize{\texttt{\textsc{Saps-QCP-easy}}}}}
\newcommand{\SAPSQCPmed}{{\footnotesize{\texttt{\textsc{Saps-QCP-med}}}}}
\newcommand{\SAPSQCPseventyfive}{{\footnotesize{\texttt{\textsc{Saps-QCP-q075}}}}}
\newcommand{\SAPSQCPninetyfive}{{\footnotesize{\texttt{\textsc{Saps-QCP-q095}}}}}
\newcommand{\SAPSSWGCPmed}{{\footnotesize{\texttt{\textsc{Saps-SWGCP-med}}}}}
\newcommand{\SAPSSWGCPseventyfive}{{\footnotesize{\texttt{\textsc{Saps-SWGCP-q075}}}}}
\newcommand{\SAPSSWGCPninetyfive}{{\footnotesize{\texttt{\textsc{Saps-SWGCP-q095}}}}}

\newcommand{\cplexregionsonehundredtiny}{\scriptsize{\texttt{\textsc{CPLEX-Regions100}}}}
\newcommand{\cplexmiktiny}{{\scriptsize{\texttt{\textsc{CPLEX-MIK}}}}}
\newcommand{\SAPSQCPtiny}{{\scriptsize{\texttt{\textsc{Saps-QCP}}}}}
\newcommand{\SPEARSWGCPtiny}{{\scriptsize{\texttt{\textsc{Spear-SWGCP}}}}}
\newcommand{\SPEARQCPtiny}{{\scriptsize{\texttt{\textsc{Spear-QCP}}}}}
\newcommand{\SAPSSWGCPtiny}{{\scriptsize{\texttt{\textsc{Saps-SWGCP}}}}}


\newcommand{\SAPSQCPmedtiny}{{\scriptsize{\texttt{\textsc{Saps-QCP-med}}}}}
\newcommand{\SAPSQCPseventyfivetiny}{{\scriptsize{\texttt{\textsc{Saps-QCP-q075}}}}}
\newcommand{\SAPSQCPninetyfivetiny}{{\scriptsize{\texttt{\textsc{Saps-QCP-q095}}}}}
\newcommand{\SAPSSWGCPmedtiny}{{\scriptsize{\texttt{\textsc{Saps-SWGCP-med}}}}}
\newcommand{\SAPSSWGCPseventyfivetiny}{{\scriptsize{\texttt{\textsc{Saps-SWGCP-q075}}}}}
\newcommand{\SAPSSWGCPninetyfivetiny}{{\scriptsize{\texttt{\textsc{Saps-SWGCP-q095}}}}}

\newcommand{\SAPSQWH}{{\footnotesize{\texttt{\textsc{Saps-QWH}}}}}
\newcommand{\SAPSQWHtiny}{{\scriptsize{\texttt{\textsc{Saps-QWH}}}}}


\newcommand{\SPEARIBMtwentyfive}{{\footnotesize{\texttt{\textsc{Spear-IBM-q025}}}}}
\newcommand{\SPEARIBMmed}{{\footnotesize{\texttt{\textsc{Spear-IBM-med}}}}}
\newcommand{\SPEARSWVmed}{{\footnotesize{\texttt{\textsc{Spear-SWV-med}}}}}
\newcommand{\SPEARSWVseventyfive}{{\footnotesize{\texttt{\textsc{Spear-SWV-q075}}}}}
\newcommand{\SPEARSWVninetyfive}{{\footnotesize{\texttt{\textsc{Spear-SWV-q095}}}}}

\newcommand{\SPEARIBMtwentyfivetiny}{{\scriptsize{\texttt{\textsc{Spear-IBM-q025}}}}}
\newcommand{\SPEARIBMmedtiny}{{\scriptsize{\texttt{\textsc{Spear-IBM-med}}}}}
\newcommand{\SPEARSWVmedtiny}{{\scriptsize{\texttt{\textsc{Spear-SWV-med}}}}}
\newcommand{\SPEARSWVseventyfivetiny}{{\scriptsize{\texttt{\textsc{Spear-SWV-q075}}}}}
\newcommand{\SPEARSWVninetyfivetiny}{{\scriptsize{\texttt{\textsc{Spear-SWV-q095}}}}}



%%%%%%%%%%%%%%%%%%%%%%%%%%%%%%%%%%%%%%%%%%%%%%
% ABBREVIATIONS
%%%%%%%%%%%%%%%%%%%%%%%%%%%%%%%%%%%%%%%%%%%%%%
\newcommand{\etal}[0]{et al.{}}
\newcommand{\eg}[0]{\emph{e.{}g.{}}}
\newcommand{\ie}[0]{\emph{i.{}e.{}}}
\newcommand{\adhoc}[0]{\emph{ad hoc}}
\newcommand{\cf}[0]{cf.{}}
\newcommand{\vs}[0]{\emph{vs}}
\newcommand{\wrt}[0]{w.{}r.{}t.{}}
\newcommand{\wrtl}[0]{with respect to}
\newcommand{\iid}[0]{i.{}i.{}d.{}}
\newcommand{\naive}[0]{na\"ive}
\newcommand{\Naive}[0]{Na\"ive}

%Math
\newcommand{\vTheta}{{\bm{\Theta}}}
\newcommand{\vtheta}{{\bm{\theta}}}
\newcommand{\vo}{{\bm{o}}}
\newcommand{\calM}{\mbox{${\cal M}$}}
\newcommand{\calD}{\mbox{${\cal D}$}}
\newcommand{\gauss}{\mbox{${\cal N}$}}
\newcommand\transpose{{\textrm{\tiny{\sf{T}}}}}

%% keep figures from going onto a page by themselves
\renewcommand{\topfraction}{0.9}
\renewcommand{\textfraction}{0.1}
\renewcommand{\floatpagefraction}{0.85}


%%%%%%%%%%%%%%%%%%%%%%%%%%%%%%%%%%%%%%%%%%%%%%
% SET UP CHANGEBAR
%%%%%%%%%%%%%%%%%%%%%%%%%%%%%%%%%%%%%%%%%%%%%%

\usepackage[outerbars,color]{changebar}
%\ifx\pdfoutput\undefined
%\else\ifnum\pdfoutput>0
%\usepackage{pdfcolmk}
%\fi\fi

\setcounter{changebargrey}{60}
\cbcolor{red}

\newcommand{\cb}[1]{\cbstart{#1} \cbend}
%\renewcommand{\cb}[1]{#1}
\newcommand{\cbgreen}[1]{\cbstart{\textcolor{green}{#1}} \cbend}
\newcommand{\tred}[1]{\textcolor{red}{#1}}


\newcommand{\denselist}{\itemsep -1.5pt\partopsep -20pt}



%%%%%%%%%%%%%%%%%%%%%%%%%%%%%%%%%%%%%%%%%%%%%%
% CONVENIENT COMMANDS for formulae
%%%%%%%%%%%%%%%%%%%%%%%%%%%%%%%%%%%%%%%%%%%%%%

\newcommand{\nangbra}[1]{$\langle{}$#1$\rangle$}
\newcommand{\angbra}[1]{\langle{}#1\rangle}
\newcommand{\bra}[1]{\left(#1\right)}
\newcommand{\squbra}[1]{\left[#1\right]}
\newcommand{\vctm}[1]{\boldmath{$#1$}\unboldmath{}}
\newcommand{\vct}[1]{\mbox{\boldmath{$#1$}\unboldmath{}}}
%\newcommand{\qed}{~\vspace*{-0.5cm}\hspace*{\textwidth}\qedsymbol~\vspace*{0.5cm}}


%%%%%%%%%%%%%%%%%%%%%%%%%%%%%%%%%%%%%%%%%%%%%%
% COMMANDS for algorithm2e.
%%%%%%%%%%%%%%%%%%%%%%%%%%%%%%%%%%%%%%%%%%%%%%
\SetCommentSty{textit}
\SetKwComment{fhcomment}{// ===== }{}
\newcommand{\nlcom}[1]{{\BlankLine\footnotesize{\fhcomment{\CommentSty{#1}}}}}
\newcommand{\com}[1]{{\footnotesize{\CommentSty{// #1}}}}

\newcommand{\fhem}[1]{\text{\emph{#1}}}
\newcommand{\algo}[2]{#1\;\small #2}

%\newcommand{\Input}[1]{Input: #1}
%\newcommand{\Output}[1]{Output: #1}
%\newcommand{\Effect}[1]{Effect: #1}

\SetKwBlock{Procedure}{begin}{end}
\SetKwFunction{Term}{TerminationCriterion()}
\SetKwFunction{Init}{GenerateInitialSolution}
\SetKwFunction{Acc}{AcceptanceCriterion}
\SetKwFunction{Best}{best}
\SetKwFunction{IteratedLocalsearch}{IteratedLocalsearch}
\SetKwFunction{Localsearch}{LocalSearch}
\SetKwFunction{IterativeFirstImprovement}{IterativeFirstImprovement}
\SetKwFunction{Pert}{Perturbation}
%\DontPrintSemicolon
\SetFuncSty{textit}

%%%%%%%%%%%%%%%%%%%%%%%%%%%%%%%%%%%%%%%%%%%%%%
% COMMANDS for theorems, lemmas, definitions, etc.
%%%%%%%%%%%%%%%%%%%%%%%%%%%%%%%%%%%%%%%%%%%%%%

%\usepackage{amsmath, amsthm, amssymb}
\newtheorem{thm}{Theorem}%[section]
\newtheorem{ex}{Example}%[section]
\newtheorem{lem}[thm]{Lemma}
\newtheorem{cor}[thm]{Corollary}
\newtheorem{obs}[thm]{Observation}

%\theoremstyle{definition}
\newtheorem{define}[thm]{Definition}
\hyphenation{ge-ne-ral-ize}

%%%%%%%%%%%%%%%%%%%%%%%%%%%%%%%%%%%%%%%%%%%%%%
% FIGURES
%%%%%%%%%%%%%%%%%%%%%%%%%%%%%%%%%%%%%%%%%%%%%%

\newcommand{\largequadraticgraph}[1]{
%        \includegraphics[width=3.9cm,height=3.9cm]{#1}
%        \includegraphics[width=4.6cm,height=4.6cm]{#1}
%        \includegraphics[width=4.9cm,height=4.6cm]{#1}
%        \includegraphics[width=5.4cm,height=5.4cm]{#1}
        \includegraphics[width=5.4cm,height=5.4cm]{#1}
  }


\newcommand{\smallquadraticgraph}[1]{
%        \includegraphics[width=3.9cm,height=3.9cm]{#1}
%        \includegraphics[width=4.6cm,height=4.6cm]{#1}
        \includegraphics[width=4.9cm,height=4.6cm]{#1}
%        \includegraphics[width=5.4cm,height=5.4cm]{#1}
}


\newcommand{\quadraticgraph}[1]{
%        \includegraphics[width=4.1cm,height=4.1cm]{#1}
        %\includegraphics[width=4.6cm,height=4.6cm]{#1}
        \includegraphics[width=5.4cm,height=5.4cm]{#1}
        %\includegraphics[width=6.15cm,height=6.15cm]{#1}
}

\newcommand{\smallgraph}[1]{
%        \includegraphics[width=6.15cm,height=4.1cm]{#1}
        %\includegraphics[width=6.9cm,height=4.6cm]{#1}
%        \includegraphics[width=12cm,height=8cm]{#1}
\includegraphics[width=5.4cm,height=3.6cm]{#1}
%	\includegraphics[width=4.95cm,height=3.3cm]{#1}
}


\newcommand{\normalgraph}[1]{
%        \includegraphics[width=6.15cm,height=4.1cm]{#1}
        \includegraphics[width=6.9cm,height=4.6cm]{#1}
%        \includegraphics[width=12cm,height=8cm]{#1}
%\includegraphics[width=5.4cm,height=3.6cm]{#1}
%	\includegraphics[width=4.95cm,height=3.3cm]{#1}
}

\newcommand{\biggergraph}[1]{
%        \includegraphics[width=6.15cm,height=4.1cm]{#1}
        \includegraphics[width=6.9cm,height=4.6cm]{#1}
%        \includegraphics[width=12cm,height=8cm]{#1}
%\includegraphics[width=5.4cm,height=3.6cm]{#1}
%	\includegraphics[width=4.95cm,height=3.3cm]{#1}
}

\newcommand{\fig}[3]{  % input_file, label, caption
  \begin{figure}[tpb]
    \begin{center}
      \normalgraph{#1}
      \caption{\label{#2}#3}
    \end{center}
  \end{figure}
}


\newcommand{\qfig}[3]{  % input_file, label, caption
  \begin{figure}[tpb]
%hh: temporarily changed:
%  \begin{figure}[p]
    \begin{center}
      \quadraticgraph{#1}
      \caption{\label{#2}#3}
    \end{center}
  \end{figure}
}

\newcommand{\largeqfig}[3]{  % input_file, label, caption
%  \begin{figure}[tpb]
%hh: temporarily changed:
  \begin{figure}[p]
    \begin{center}
      \largequadraticgraph{#1}
      \caption{\label{#2}#3}
    \end{center}
  \end{figure}
}

\newcommand{\sides}[6]{         %fig1, fig2, cap1, cap2, cap, label
        \begin{figure*}[tbph]
        \hfill
            \subfigure[#3]
            {
                \quadraticgraph{#1}
                  }
        \hfill{}
        \hfill
            \subfigure[#4]
            {
                \quadraticgraph{#2}
                  }
        \hfill{}
        \caption{#5}
        \label{#6}
%        \vspace{-0.245cm}
        \end{figure*}
}

\newcommand{\threefig}[8]{         %fig1, fig2, fig3, cap1, cap2, cap3, cap, label
    \begin{figure*}[tbp]
      \begin{center}
        \mbox{
            \subfigure[#4]
            {
                \quadraticgraph{#1}
                  }
            \subfigure[#5]
            {
                \quadraticgraph{#2}
                  }
            \subfigure[#6]
            {
                \quadraticgraph{#3}
                  }
        }
        \vspace{-0.4cm}
        \caption{#7}
        \label{#8}
      \end{center}
      \vspace{-0.245cm}
    \end{figure*}
}

\newcommand{\threebfig}[8]{         %fig1, fig2, fig3, cap1, cap2, cap3, cap, label
    \begin{figure*}[tb]
      \begin{center}
        \mbox{
            \subfigure[#4]
            {
                \smallgraph{#1}
                  }
            \subfigure[#5]
            {
                \smallgraph{#2}
                  }
            \subfigure[#6]
            {
                \smallgraph{#3}
                  }
        }
        \vspace{-0.4cm}
        \caption{#7}
        \label{#8}
      \end{center}
      \vspace{-0.245cm}
    \end{figure*}
}

\usepackage{fullpage}
\usepackage{setspace}
\usepackage{subfiles}
\usepackage{graphicx}
%%\usepackage{hyperref}

\newcommand{\note}[1]{}
% comment the next line to turn off notes
\renewcommand{\note}[1]{~\\\frame{\begin{minipage}[c]{\textwidth}\vspace{2pt}\center{#1}\vspace{2pt}\end{minipage}}\vspace{3pt}\\}


\begin{document}

\title{Manual for SMAC version vDEV}

\author{
Frank~Hutter \& Steve~Ramage\\
Department of Computer Science\\
University of British Columbia\\
Vancouver, BC\ \ V6T~1Z4, Canada\\
\texttt{\{hutter,seramage\}@cs.ubc.ca}
}


\maketitle

\tableofcontents

\section{Introduction}\label{sec:intro}

This document is the manual for SMAC~\cite{HutHooLey11-SMAC} (an acronym for \emph{Sequential Model-based Algorithm Configuration}). SMAC aims to solve the following \emph{algorithm configuration} problem: Given a binary of a parameterized algorithm $\mathcal{A}$, a set of instances $\mathcal{S}$ of the problem $\mathcal{A}$ solves, and a performance metric $m$, find parameter settings of $\mathcal{A}$ optimizing $m$ across $\mathcal{S}$.

In slightly more detail, users of SMAC must provide:
\begin{itemize}
\item a parametric algorithm $\mathcal{A}$ (an executable to be called from
the command line), 
\item a description of $\mathcal{A}$'s parameters $\theta_1,\dots,\theta_n$ and their domains $\Theta_1, \dots, \Theta_n$, 
\item a set of benchmark instances, $\Pi$, and
\item the objective function with which to measure and aggregate algorithm preformance results.
\end{itemize}

SMAC then executes algorithm $\mathcal{A}$ with different \emph{parameter configurations} (combinations of parameters 
$\langle{}\theta_1,\dots,\theta_n\rangle{} \in \Theta_1 \times \cdots \times \Theta_n$, on instances $\pi \in \Pi$),
searching for the configuration that yields overall best performance across the benchmark instances under the supplied objective. For more details please see~\cite{HutHooLey11-SMAC}; if you use SMAC in your research, please cite that article. It would also be nice if you sent us an email -- we are always interested in additional application domains.

%%%%%%%%%%%%%%%%%%%%%%%%%%%%%%%%%%%%%%%%%%%%%%%%%%%%%%%%%%%%%%%%%%%%
\subsection{License}
%%%%%%%%%%%%%%%%%%%%%%%%%%%%%%%%%%%%%%%%%%%%%%%%%%%%%%%%%%%%%%%%%%%%

SMAC will be released under a dual usage license.  
Academic \& non-commercial usage is permitted free of charge. Please contact us to discuss commercial usage.

\subsection{System Requirements}

SMAC itself requires only Java 6 \footnote{Sun Java version 1.6.0\_23 or later recommended} or newer to run. The included scripts are currently only available for Unix-compatible operating systems. The included example scenarios require Ruby. 

\subsection{Version}
This version of the manual is for SMAC vDEV-000$\!\!$.
\\
\subfile{githashes.tex}



%\subsection{Included Example Scenarios}
%\note{FH: TODO after we decide what to include}

\section{Differences Between SMAC and ParamILS}

There are a number of differences between SMAC and ParamILS, including the following.

\begin{itemize}
\item \textbf{Support for continuous parameters:} 
While ParamILS was limited to categorical parameters, SMAC also natively handles continuous and integer parameters. See Section \ref{sec:param_decl_clauses} for details.

\item \textbf{Run objectives:} 
Not all of ParamILS's run objectives are supported at this time. If you require an unsupported objective please let us know.

\item \textbf{Order of instances:} 
In contrast to ParamILS, the order of instances in the instance file does not matter in SMAC.

\item \textbf{Configuration time budget and runtime overheads:} 
Both ParamILS and SMAC accept a time budget as an input parameter. ParamILS only keeps track of the CPU time the target algorithm reports and terminates once the sum of these runtimes exceeds the time budget; it does \emph{not} take into account overheads due to e.g.\ command line calls of the target algorithm. In cases where the reported CPU time of each target algorithm run was very small (e.g.\ milliseconds), these unaccounted overheads could actually dominate ParamILS's wall-clock time.
SMAC offers a more flexible management of its runtime overheads through the options 
\textbf{-~$\!$-use-cpu-time-in-tunertime} and \textbf{-~$\!$-wallclock-limit}. See Section \ref{sec:wall-clock} for details on the wall clock time limit.

\item \textbf{Resuming previous runs:} 
While this was not possible in ParamILS, in SMAC you can resume previous runs from a saved state.
Please refer to Section \ref{sec:state-restoration} for how to use the state restoration feature. Section \ref{subsec:state-files} describes the file format for saved states.

\item \textbf{Feature files:} 
SMAC accepts as an optional input a feature file providing additional information about the instances in the training set; see Section \ref{sec:feature_file_format}.


\item \textbf{Algorithm wrappers:} 
The wrapper syntax has been extended in SMAC to support additional results in the ``solved'' field. Specifically, there is a new result \textbf{ABORT} signalling that the configuration process should be aborted (e.g. because the wrapper is in an inconsistent state that should never be reached). A similar behaviour is triggered if option \textbf{-~$\!$-abort-on-first-run-crash} is set and the first run returns \textbf{CRASHED}. Additionally, the wrapper can also return additional data to SMAC that is associated with the run \footnote{This data will be saved in the run and results file (Section \ref{subsec:state-files}) that is used in state saving}. For more information see Section \ref{sec:wrapper_output}.

\item \textbf{Instance files vs. instance/seed files:} 
The \textbf{instance\_file} parameter now auto-detects whether the file conforms to ParamILS's \textbf{instance\_file} or \textbf{instance\_seed\_file} format. SMAC treats the latter option as an alias for the former. See Section \ref{sec:instance_file_format} for details.
While SMAC is backwards compatible with previous (space-separated) files, the preferred format is now \texttt{.csv}.
\end{itemize}

  
\section{Commonly Used Options}
\documentclass[manual.tex]{subfiles}\begin{document}
\subsection{Running SMAC}
To get started with an existing configuration scenario you simply
need to execute smac as follows:

\begin{verbatim}
./smac --scenarioFile <file> --numRun 0
\end{verbatim}

This will execute SMAC with the default options on the scenario specified in the file. 
Some commonly-used non-default options of SMAC are described in this section. The \textbf{-~$\!$-numRun} argument controls the seed and names of output files (to support parallel independent runs)

\subsection{Testing the Wrapper} \label{sec:test}
SMAC includes a method of Testing Algorithm Execution, via the \texttt{smac-algotest} utility. It takes the required scenario options \footnote{Unfortunately it cannot read scenario files currently}
 \textbf{-$~\!$-execDir}, \textbf{-$~\!$-paramFile}, \textbf{-$~\!$-algo}, \textbf{-$~\!$-cutoff\_time} the instance, and configuration to run on.

For example:
\begin{verbatim} 
./smac-algotest --execDir <dir> --paramFile <file>  --algo <file> 
--cutoff_time 300 --instance <instance>  --config <config string>
 -P[name]=[value] -P[name]=[value]...
\end{verbatim}

Some parameters deserve special mention:
\begin{enumerate}
\item The config string syntax is a single string with ``-name=`value' '' ... you can also specify \texttt{RANDOM} which will generate a random configuration or \texttt{DEFAULT} which will generate  the default configuration.

\item The \texttt{-P} parameters are optional and allow overriding specific values in the configuration (this is useful primarily for RANDOM and DEFAULT, to allow you to set certain values). To set the \texttt{sort\-algo} to \texttt{merge} you would specify \texttt{\-Psort\-algo=merge}.
\end{enumerate}


\subsection{ROAR Mode}

\begin{verbatim}
./smac --scenarioFile <file> --executionMode ROAR --numRun 0
\end{verbatim}

This will execute the ROAR algorithm, a special case of SMAC that uses an empty model and random selection of 
configurations. See \cite{HutHooLey11-SMAC} for details on ROAR.

\subsection{Adaptive Capping}
\begin{verbatim}
./smac --scenarioFile <file> --adaptiveCapping true --numRun 0
\end{verbatim}
Adaptive Capping (originally introduced for ParamILS~\cite{ParamILS-JAIR}, but also applicable in SMAC~\cite{HutHooLey11-censoring}) will cause SMAC to only schedule algorithm runs for as long as is needed to determine whether they are better than the current incumbent. Without this option, each target algorithm runs up to the runtime specified in the configuration scenario file \textbf{-~$\!$-cutoffTime}.

\noindent{}\textsc{Note:} Adaptive Capping should only be used when the \textbf{-~$\!$-runObj} is RUNTIME.
Adaptive capping can drastically improve SMAC's performance for scenarios with a large difference between 
\textbf{-~$\!$-cutoffTime} and the runtime of the best-performing configurations.
Related configuration options are \textbf{-~$\!$-capSlack}, \textbf{-~$\!$-capAddSlack}, and \textbf{-~$\!$-imputationIterations}.

\subsection{Wall-Clock Limit}\label{sec:wall-clock}
\begin{verbatim}
./smac --scenarioFile <file> --runtimeLimit <seconds> --numRun 0
\end{verbatim}
SMAC offers the option to terminate after using up a given amount of wall-clock time. This option is useful to limit the overheads of starting target algorithm runs, which are otherwise unaccounted for.
This option does not override \textbf{-~$\!$-tunerTimeout} or other options that limit the duration of the configuration run; whichever termination criterion is reached first triggers termination. 

\subsection{Change Initial Incumbent}\label{sec:initial-incumbent}
\begin{verbatim}
./smac --scenarioFile <file> --numRun 0 --initialIncumbent <config string>
\end{verbatim}

SMAC offers the option to specify the initial incumbent, and by default uses the default configuration specified in the parameter file. The argument to \textbf{-~$\!$-initialIncumbent} follows the same conventions as in Section \ref{sec:test}.

\subsection{State Restoration}\label{sec:state-restoration}
\begin{verbatim}
./smac --scenarioFile <file> --restoreStateFrom <dir> 
       --restoreIteration <iteration> --numRun 0
\end{verbatim}
SMAC will read the files in the specified directory and restore 
its state to that of the saved SMAC run at the specified iteration.
Provided the remaining options (e.g. \textbf{-~$\!$-seed}, \textbf{-~$\!$-overall\_obj}) are set identicially, SMAC should continue along the same trajectory.

This option can also be used to restore runs from SMAC v1.xx (although due to the lossy nature of Matlab files and differences in random calls you will not get the same resulting trajectory). By default the state can be restored to iterations that are powers of 2, as well as the 2 iterations prior to the original SMAC run stopping. 
If the original run crashed, additional information is saved, often allowing you to replay the crash.

\textsc{Note:} When you restore a SMAC state, you are in essence preloading a set of runs and then running the scenario. In certain cases, if the scenario has been changed in the meantime, this may result in undefined behaivor. Changing something like \textbf{-~$\!$-tunerTimeout} is usually a safe bet, however changing something central (such as \textbf{-~$\!$-runObj}) would not be.

To check the available iterations that can be restored from a saved
directory, use:
\begin{verbatim}
./smac-possible-restores <dir>
\end{verbatim}

To disable saving any state information to disk, use 
\begin{verbatim}
./smac --scenarioFile <file> --stateSerializer NULL --numRun 0
\end{verbatim}


\subsection{Concurrent Algorithm Execution Requests}
\begin{verbatim}
./smac --scenarioFile <file> --maxConcurrentAlgoExecs <num> --numRun 0
\end{verbatim}

In certain circumstances, it may be much faster to allow more than one target algorithm execution at once,
(e.g., when multiple cores are available or when actual algorithm execution is I/O bound). 
To exploit this, you can have SMAC schedule multiple runs at a time. If \textbf{-~$\!$-adaptiveCapping} is not set, this will result in the same trajectory as a sequential version (when \textbf{-~$\!$-maxConcurrentAlgoExecs} is set to 1). When adaptive capping is enabled, concurrent runs are scheduled with cutoff times as if each were the first of the runs to be scheduled.


\subsection{Named Rungroups}
\begin{verbatim}
./smac --scenarioFile <file> --runGroupName <foldername> --numRun 0
\end{verbatim}
All output is written to the folder \texttt{$<$foldername$>$}; runs differing in \textbf{-~$\!$-numRun} will yield different output files in that folder.


\subsection{Offline Validation}

SMAC includes a tool for the offline assessment of incumbents selected during the configuration process.
By default, given a test instance file with $N$ instances, SMAC performs $\approx$ 1\,000 target algorithm validation runs per configuration (rounded up to the nearest multiple of N).

By default, SMAC limits the number of seeds used in validation runs to 1\,000 seeds per instance. This number can be changed as in the following example:
\begin{verbatim}
./smac --scenarioFile <file> --numSeedsPerTestInstance 50
\end{verbatim}
(This parameter does not have any effect in the case of instance/seed files.)


\subsubsection{Limiting the Number of Instances Used in a Validation Run}

To use only some of the instances or instance seeds specified you can limit them with the \textbf{-~$\!$-numTestInstances} parameter. When this parameter is specified, SMAC will only use the specified number of lines from the top of the file, and will keep repeating them until enough seeds are used: 
\begin{verbatim}
./smac --scenarioFile <file> --numTestInstances 10
\end{verbatim}
For instance files containing seeds, this option will only use the specified number of instance seeds in the file.

\subsubsection{Disabling Validation}
Validation can be skipped alltogether as follows:
\begin{verbatim}
./smac --scenarioFile <file> --skipValidation
\end{verbatim}

\subsubsection{Standalone Validation}
SMAC also includes a method of validating configurations outside of a smac run.
You can supply a configuration using the \textbf{-~$\!$-configuration} option. All scenario options are applicable to the standalone validator, but check the usage screen to see all the options available \textsc{NOTE:} Some options while present are not applicable for validation but are presented anyway.

Here is an example call:
\begin{verbatim}
./smac-validate --scenarioFile <file> --numValidationRuns 10000 
     --configuration <config string> --maxConcurrentAlgoExecs 8 --numRun 0
\end{verbatim}
%
Usage notes for the offline validation tool:
\begin{enumerate}
\item This validates against the test set only; the training instance set is not used.
\item By default this outputs into the current directory; you can change the output directory with the option \textbf{-~$\!$-runGroupName}.
\item You can also validate against a trajectory file issued by \textbf{-$~\!$-trajectoryFile} option. 


\end{enumerate}




\end{document}

%%%%%%%%%%%%%%%%%%%%%%%%%%%%%%%%%%%%%%%%%%% 

\section{File Format Reference} 
%%%%%%%%%%%%%%%%%%%%%%%%%%%%%%%%%%%%%%%%%%%
\subfile{file-formats}

%%%%%%%%%%%%%%%%%%%%%%%%%%%%%%%%%%%%%%%%%%%%%%%%%%%%%%%%%%%%%%%%%%%%
\section{Interpreting SMAC's Output}
%%%%%%%%%%%%%%%%%%%%%%%%%%%%%%%%%%%%%%%%%%%%%%%%%%%%%%%%%%%%%%%%%%%%


\subfile{output}
%%\documentclass[manual.tex]{subfiles} 
\begin{document}

%%%%%%%%%%%%%%%%%%%%%%%%%%%%%%%%%%%%%%%%%%%%%%%%%%%%%%%%%%%%%%%%%%%%
\subsection{Option Files}
%%%%%%%%%%%%%%%%%%%%%%%%%%%%%%%%%%%%%%%%%%%%%%%%%%%%%%%%%%%%%%%%%%%%
Option Files are a way of saving a different set of values 
frequently used with SMAC without having to specify them on every execution. 
The general format for an option file is the name
of the configuration option (without the two dashes), an equal sign, and then the value (for booleans it should be true or false, lowercase). Currently options that take multiple arguments are not supported. Additionally you can not use aliases that are single dashed (\eg{ to override the Experiment Directory, you must use \textbf{-~$\!\!$-experiment-dir} and not \textbf{-e}})  

When using Option Files it is important that no two files (including the Scenario File), specify the same option, the resulting configuration is undefined, and in general this will not throw an error.

%%%%%%%%%%%%%%%%%%%%%%%%%%%%%%%%%%%%%%%%%%%%%%%%%%%%%%%%%%%%%%%%%%%%
\subsubsection{Scenario File}
%%%%%%%%%%%%%%%%%%%%%%%%%%%%%%%%%%%%%%%%%%%%%%%%%%%%%%%%%%%%%%%%%%%%

The Scenario Option File, or Scenario File, is the recommended way of configuring SMAC \footnote{Nothing in general prevents you from specifying non-scenario options in these files, but in general you should restrict your files to these.}.
The Scenario Files used in SMAC are backwards compatible with ParamILS and the name of option names here reflect that\footnote{Every option name listed here is in fact an alias for an existing option listed in the section \ref{sec:options-ref} and it is entirely possible to use SMAC without using Scenario Files.}. \textsc{Note:} \textbf{cutoff\_length} is not currently supported.

\begin{description}
\item [algo] An algorithm executable or wrapper script around
an algorithm that conforms with the input/output format specified
in section \ref{sec:exec-spec}. The string here should be invokable via the system shell.
\item [{execdir}] Directory to execute \texttt{<algo>} from: (\ie{ ``cd
\texttt{<execdir>}; \texttt{<algo>}'' })
\item [{deterministic}] A boolean that governs whether or not the algorithm should be treated as deterministic. For backwards compatibility with ParamILS, this option also supports using 0 for false, and 1 for true. SMAC will never invoke the target algorithm more than once for any given instance, seed and configuration. If this is set to \texttt{true}, SMAC will never invoke the target algorithm more than once for any given instance and configuration.

\item [{run\_obj}] Determines how to convert the resulting output line (as defined in Section
 \ref{sec:wrapper_output}) into a scalar quantifying how ``good'' a single algorithm execution
is, (\eg{ how long it took to execute, how good of a solution it found,
etc...}). SMAC will attempt to \emph{minimize} this objective. 

 Currently implemented objectives are the following:
\item [{
\begin{tabular}{|c|c|}
\hline 
Name & Description\tabularnewline
\hline 
\hline 
RUNTIME & Minimize the reported runtime of the algorithm.\tabularnewline
\hline 
QUALITY & Minimize the reported quality of the algorithm.\tabularnewline
\hline 
\end{tabular}}]~
\item [{overall\_obj}] While \textbf{run\_obj} defines the objective function
for a single algorithm run, \textbf{overall\_obj} defines how those
single objectives are combined to reach a single scalar value to compare
two parameter configurations. Implemented examples for this are as
follows:
\item [{%
\begin{tabular}{|c|c|}
\hline 
Name & Description\tabularnewline
\hline 
\hline 
MEAN & The mean of the values\tabularnewline
\hline
MEAN1000 & Unsuccessful runs are counted as 1000 $\times$ \textbf{target\_run\_cputime\_limit}\tabularnewline
\hline 
MEAN10 & Unsuccessful runs are counted as 10 $\times$ \textbf{target\_run\_cputime\_limit}\tabularnewline
\hline 
\hline 
\end{tabular}}]~
\item [{target\_run\_cputime\_limit}] The CPU time after which a single algorithm execution
will be terminated as unsuccess (and treated as a \textbf{TIMEOUT}). This is an important parameter: If chosen too high, lots of time will be wasted with unsuccessful
runs. If chosen too low the optimization is biased to perform well
on easy instances only. 
%\item [{cutoff\_length}] The runlength after which a single algorithm execution
%will be terminated unsuccesfully. The actual semantic meaning of this
%value is up to the target algorithm. NOT SUPPORTED IN THIS VERSION)
\item [{cputime\_limit}] The limit of the CPU time allowed for configuration (\ie{The sum of all algorithm runtimes, and by default the sum of the CPU time of SMAC itself}).
\item [{wallclock\_limit}] The limit of the amount of wallclock (or real) time allowed for configuration. 
\item [{paramfile}] Specifies the file with the parameters of the algorithm.
The format of this file is covered in Section \ref{sec:paramfile}.
\item [{outdir}] Specifies the directory SMAC should write its results
to. 
\item [{instance\_file}] Specifies the file containing the list of problem instances (and possibly seeds) for SMAC to use during the \emph{Automatic Configuration Phase}. The ParamILS parameter \textbf{instance\_seed\_file} aliases this one and the format is auto-detected. The format of these files is covered in section \ref{sec:instance_file_format}.
\item [{test\_instance\_file}] Specifies the file containing the list of problem instances (and possibly seeds) for SMAC to use during \emph{Validation Phase}. The ParamILS parameter \textbf{test\_instance\_seed\_file} aliases this one and the format is auto-detected. The format of these files is covered in section \ref{sec:instance_file_format}.
\item [{feature\_file}] Specifies the a file with the features for the instances in the \textbf{instance\_file} and possibly the \textbf{test\_instance\_file} \footnote{The Validator will load features into memory for test instances if they exist.}. The format of this file is covered in section \ref{sec:feature_file_format}.

\end{description}

%%%%%%%%%%%%%%%%%%%%%%%%%%%%%%%%%%%%%%%%%%%%%%%%%%%%%%%%%%%%%%%%%%%%
\subsection{Instance File Format} \label{sec:instance_file_format}
%%%%%%%%%%%%%%%%%%%%%%%%%%%%%%%%%%%%%%%%%%%%%%%%%%%%%%%%%%%%%%%%%%%%


The files used by the \textbf{instance\_file} \& \textbf{test\_instance\_file} options 
come in four potential formats, all of which are CSV based\footnote{Specifically each cell should be double-quoted (\ie{''}), and use a comma as a cell delimiter. SMAC also supports the old method of reading files that use space as a cell delimiter and do not enclose values. However these files cannot handle \textbf{Instance Name}'s that contain spaces.}. Before specifying the formats it is important to note the three kinds of information that are specified with instances \footnote{Features which are required for SMAC but not ParamILS are specified in a seperate file see section \ref{sec:feature_file_format}.}.

\begin{description}
\item[Instance Name] The name of the instance that was selected. This should be meaningful to the target algorithm we are configuring \footnote{Generally \textbf{Instance Names} reference specific files on disk.}.
\item[Instance Specific Information] A free form text string (with no spaces or line breaks) that will be passed to the Target Algorithm whenever executed.
\item[Seed] A specific seed to use when executing the target algorithm.
\end{description}

The possible formats are as follows, and depend on what information you'd like to specify.

\begin{enumerate}
\item	Each line specifies only a unique \textbf{Instance Name}. No \textbf{Instance Specific Information} will be used, and \textbf{Seed}'s will be automatically generated.

\item  Each line specifies a \textbf{Seed} followed by the \textbf{Instance Name}. Every line must be unique, but for each \textbf{Instance Name} additional seeds will be used in order, when that instance is selected.

\item Each line specifies a \textbf{Instance Name} followed by the \textbf{Instance Specific Information}. Every \textbf{Instance Name} must be unique, \textbf{Seed}'s will be automatically generated.

\item Each line specifies a \textbf{Seed} followed by the \textbf{Instance Name} followed by the \textbf{Instance Specific Information}. Every line must be unique, and furthermore, for all \textbf{Instance Name}'s the \textbf{Instance Specific Information} must be the same for all \textbf{Seed} values (\ie{You cannot specify different instance specific information that is a function of the seed used}).

\end{enumerate}

%%%%%%%%%%%%%%%%%%%%%%%%%%%%%%%%%%%%%%%%%%%%%%%%%%%%%%%%%%%%%%%%%%%%
\subsection{Feature File Format}\label{sec:feature_file_format}
%%%%%%%%%%%%%%%%%%%%%%%%%%%%%%%%%%%%%%%%%%%%%%%%%%%%%%%%%%%%%%%%%%%%



The \textbf{feature\_file} specifies features that are to be used for instances. Feature Files are specified in CSV format, the first column of every row should list an \textbf{Instance Name} as it appears in the \textbf{instance\_file}. The subsequent columns should list \texttt{double} values specifying a computed continuous feature. By convention the value $-512$, and $-1024$ are used to signify that a feature value is missing or not applicable. All instances must have the same number of features.

At the top of the file there \textsc{must} appear a header row, the cell that appears above the instance names is unimportant, but for each feature a unique and \emph{non-numeric} (\ie{ it must contain atleast one letter}) feature name must be specified.

%%%%%%%%%%%%%%%%%%%%%%%%%%%%%%%%%%%%%%%%%%%%%%%%%%%%%%%%%%%%%%%%%%%%
\subsection{Parameter Configuration Space Format} \label{sec:paramfile}
%%%%%%%%%%%%%%%%%%%%%%%%%%%%%%%%%%%%%%%%%%%%%%%%%%%%%%%%%%%%%%%%%%%%

 \subfile{pcs-subfile}


\end{document}


 
\section{Developer Reference}

\subfile{dev-ref}

%%%%%%%%%%%%%%%%%%%%%%%%%%%%%%%%%%%%%%%%%%%%%%%%%%%%%%%%%%%%%%%%%%%%
\section{Acknowledgements}
%%%%%%%%%%%%%%%%%%%%%%%%%%%%%%%%%%%%%%%%%%%%%%%%%%%%%%%%%%%%%%%%%%%%

We are indebted to Jonathan Shen for porting our random forest code from C to Java in preparation for a Java port of all of SMAC. We thank Marius Schneider for many valuable bug reports and suggestions for improvements. Thanks also to Chris Thornton for being a secondary beta tester.


\renewcommand{\bibsection}{\section{References}}
\bibliographystyle{apalike}
\bibliography{short,frankbib}


\section{Appendix}
\subsection{Return Codes}

\begin{table}[h]
\begin{tabular}{| c | c | c |}
\hline
Value & Error Name & Description \\
\hline
\hline
0 & \textbf{Success} & Everything completed successfully \\
\hline
1 & \textbf{Parameter Error} & There was a problem with the input arguments or files  \\
\hline
2 & \textbf{Trajectory Divergence} & For some reason SMAC has taken a unexpected path \\

& & (\eg{} SMAC executes a run that does not match a run \\
& & in the \textbf{-~$\!\!$-runHashCodeFile}) \\
\hline
3 & \textbf{Serialization Exception} & A problem occurred when saving or restoring state \\
\hline
255 & \textbf{Other Exceptions} & Some other error occurred \\
\hline
\end{tabular}
\end{table}

\textsc{NOTE:} All error conditions besides 255 are fixed. However in future some exceptions that previously reported 255 may be changed to a non 255 value as needed / requested

\subsection{Version History of Java SMAC}
	\begin{description}
		\item[Version 2.00 (Aug-2012)] First Internal Release of Java SMAC (this is a port and extension of the original Matlab version).
		\item[Version 2.02 (Oct-2012)] First Public Release of SMAC v2 and contained many fixes from the previous release.
		\item[Version 2.04 (TBD)] Second Release of Java SMAC including the following improvements:
			\begin{enumerate}
			\item Validation file output times consistent with Tuner Times
			\item Some \textbf{INFO} log statements have been moved to \textbf{DEBUG} and some \textbf{DEBUG} to \textbf{TRACE}
			\item Added support for verifying whether responses of SAT and UNSAT are consistent with Instance Specific Information see \textbf{$-~\!-$verifySAT} option for more information
			\end{enumerate}
		
	\end{description}
\subsection{Known Issues}

\begin{enumerate}
\item SMAC may crash with a \texttt{ConvergenceException} when \textbf{-$~\!$-adaptiveCapping} is \texttt{true}, this has happened only once out of thousands of Runs, but we have no fix at this time other than trying a different seed.
\item Trajectory file does not output standard deviation
\item Using any alias for -$~\!$-showHiddenParameters, -$~\!$-help, or -$~\!$-version as values to other arguments (\eg{ Setting -$~\!$-runGroupName -$~\!$-help}) does not parse correctly (This is unlikely to be fixed until someone complains).
\item Using large parameter values in continuous integral parameters, may cause loss of precision, and or crashes if the values are too big.
\item On older versions of Java ($<$1.6.0\_23), SMAC  may get an IOException with Out Of Memory when trying to execute the target algorithms
\end{enumerate}

\subsection{Options Reference}
\label{sec:options-ref}
\subfile{options-ref}

\end{document}

