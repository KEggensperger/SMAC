\documentclass[11pt,letterpaper,oneside]{article}
%%%%%%%%%%%%%%%%%%%%%%%%%%%%%%%%%%%%%%%%%%%%%%
% GENERAL PACKAGES
%%%%%%%%%%%%%%%%%%%%%%%%%%%%%%%%%%%%%%%%%%%%%%

%\usepackage{epsfig}
\usepackage{subfigure}
\usepackage{url}
\urlstyle{sf}	% typeset urls in sans-serif

%\usepackage{amsmath}
%\usepackage{amsfonts}
%\usepackage{color}
%\usepackage{framed}

%\usepackage{makeidx}  % allows for indexgeneration
\usepackage{bm}
\usepackage{graphicx}
\usepackage[boxed, algoruled, vlined, linesnumbered]{algorithm2e} % noresetcount
\usepackage{times}
\usepackage{dsfont}
\usepackage[longnamesfirst,numbers]{natbib} %FH: longnamesfirst does not seem to work :(
\usepackage{microtype}
%\usepackage[longnamesfirst]{natbib}
\usepackage{amsmath, amssymb, amsfonts}

%Next 7 lines: tell natbib not to put bibliography onto new page
\makeatletter
\renewcommand\bibsection%
{
  \section*{\refname
    \@mkboth{\MakeUppercase{\refname}}{\MakeUppercase{\refname}}}
}
\makeatother

%% Define a new style for URLs that will use a smaller font.
%% use a different one for footnotes (requires manual switching)
\makeatletter
\def\url@newstyle{%
  \@ifundefined{selectfont}{\def\UrlFont{\sf}}{\def\UrlFont{\small}}}
\def\url@newFNstyle{%
  \@ifundefined{selectfont}{\def\UrlFont{\sf}}{\def\UrlFont{\scriptsize}}}
\makeatother
\urlstyle{new}


\newcommand{\todo}[1]{\textbf{TODO: #1}}
\newcommand{\todocrc}[1]{\note{TODOCRC: #1}}
\newcommand{\cmnt}[1]{\textbf{COMMENT: #1}}
\newcommand{\fhcrc}[1]{\note{FH CRC comment: #1}}


\newcommand{\SingleInstSaps}{{\small{\texttt{\textsc{Single\-Inst\-Saps}}}}}
\newcommand{\SingleInstSpear}{{\small{\texttt{\textsc{Single\-Inst\-Spear}}}}}
\newcommand{\Broad}{{\small{\texttt{\textsc{Broad}}}}}
\renewcommand{\AlTitleFnt}[1]{{\bf{}#1}}

%%%%%%%%%%%%%%%%%%%%%%%%%%%%%%%%%%%%%%%%%%%%%%
% CONVENIENT COMMANDS for commenting text
%%%%%%%%%%%%%%%%%%%%%%%%%%%%%%%%%%%%%%%%%%%%%%
\newcommand{\hide}[1]{}
\newcommand{\fh}[1]{{\bf{}FH: #1}}
\newcommand{\hh}[1]{\note{HH says: #1}}
\renewcommand{\hh}[1]{}
\newcommand{\klb}[1]{{\bf{}KLB: #1}}



%Configurators
\newcommand{\algofont}[1]{{\footnotesize{\textsc{#1}}}}
\newcommand{\paramils}{\algofont{Param\-ILS}}
%\newcommand{\paramils}{\textsc{Param\-ILS}}
\newcommand{\focusedils}{\algofont{Focused\-ILS}}
\newcommand{\basicils}{\algofont{Basic\-ILS}}
\newcommand{\gga}{\algofont{GGA}}
\newcommand{\smac}{\algofont{SMAC}}
\newcommand{\roar}{\algofont{ROAR}}
\newcommand{\tbspo}{\algofont{TB-SPO}}
\newcommand{\spop}{{\algofont{SPO$^+$}}}
\newcommand{\randomstar}{\algofont{Ran\-dom$^*$}}
\newcommand{\randomsearchstar}{\randomstar}
\newcommand{\frace}{{\algofont{F-Race}}}

\newcommand{\salgofont}[1]{{\scriptsize{\textsc{#1}}}}
\newcommand{\sgga}{\salgofont{GGA}}
\newcommand{\ssmac}{\salgofont{SMAC}}
\newcommand{\sroar}{\salgofont{ROAR}}
\newcommand{\stbspo}{\salgofont{TB-SPO}}
\newcommand{\sspop}{{\salgofont{SPO$^+$}}}

%SAT solvers
\newcommand{\spear}{\algofont{SPEAR}}
\newcommand{\saps}{\algofont{SAPS}}
\newcommand{\satenstein}{\algofont{SATenstein}}

%MIP solvers
\newcommand{\lpsolve}{\algofont{lp\-solve}}
\newcommand{\gurobi}{\algofont{Gu\-ro\-bi}}
\newcommand{\cplex}{\algofont{CPLEX}}
\newcommand{\ibmcplex}{\algofont{IBM ILOG CPLEX}}

%Instance sets
\newcommand{\shoe}{\textsc{SHOE}}
%\newcommand{\MASS}{\textsc{MASS}}
\newcommand{\QCP}{QCP}
\newcommand{\SWGCP}{SWGCP}
\newcommand{\MASS}{MASS}
\newcommand{\MIK}{MIK}
\newcommand{\CLS}{CLS}
\newcommand{\MJA}{MJA}
\newcommand{\corlat}{\textsc{CORLAT}}
\newcommand{\regionsonehundred}{\textsc{Regions100}}
\newcommand{\regionstwohundred}{\textsc{Regions200}}
\newcommand{\regionsseventy}{\textsc{Regions70}}

\newcommand{\SPEARSWGCP}{{\footnotesize{\texttt{\textsc{Spear-SWGCP}}}}}
\newcommand{\SPEARQCP}{{\footnotesize{\texttt{\textsc{Spear-QCP}}}}}

\newcommand{\SAPSSWGCP}{{\footnotesize{\texttt{\textsc{Saps-SWGCP}}}}}
\newcommand{\SAPSSWGCPhomog}{{\footnotesize{\texttt{\textsc{Saps-SW-hom}}}}}

\newcommand{\cplexregionsonehundred}{{\footnotesize{\texttt{\textsc{CPLEX-Regions100}}}}}
\newcommand{\cplexmik}{{\footnotesize{\texttt{\textsc{CPLEX-MIK}}}}}
\newcommand{\SAPSQCP}{{\footnotesize{\texttt{\textsc{Saps-QCP}}}}}
\newcommand{\SAPSQCPeasy}{{\footnotesize{\texttt{\textsc{Saps-QCP-easy}}}}}
\newcommand{\SAPSQCPmed}{{\footnotesize{\texttt{\textsc{Saps-QCP-med}}}}}
\newcommand{\SAPSQCPseventyfive}{{\footnotesize{\texttt{\textsc{Saps-QCP-q075}}}}}
\newcommand{\SAPSQCPninetyfive}{{\footnotesize{\texttt{\textsc{Saps-QCP-q095}}}}}
\newcommand{\SAPSSWGCPmed}{{\footnotesize{\texttt{\textsc{Saps-SWGCP-med}}}}}
\newcommand{\SAPSSWGCPseventyfive}{{\footnotesize{\texttt{\textsc{Saps-SWGCP-q075}}}}}
\newcommand{\SAPSSWGCPninetyfive}{{\footnotesize{\texttt{\textsc{Saps-SWGCP-q095}}}}}

\newcommand{\cplexregionsonehundredtiny}{\scriptsize{\texttt{\textsc{CPLEX-Regions100}}}}
\newcommand{\cplexmiktiny}{{\scriptsize{\texttt{\textsc{CPLEX-MIK}}}}}
\newcommand{\SAPSQCPtiny}{{\scriptsize{\texttt{\textsc{Saps-QCP}}}}}
\newcommand{\SPEARSWGCPtiny}{{\scriptsize{\texttt{\textsc{Spear-SWGCP}}}}}
\newcommand{\SPEARQCPtiny}{{\scriptsize{\texttt{\textsc{Spear-QCP}}}}}
\newcommand{\SAPSSWGCPtiny}{{\scriptsize{\texttt{\textsc{Saps-SWGCP}}}}}


\newcommand{\SAPSQCPmedtiny}{{\scriptsize{\texttt{\textsc{Saps-QCP-med}}}}}
\newcommand{\SAPSQCPseventyfivetiny}{{\scriptsize{\texttt{\textsc{Saps-QCP-q075}}}}}
\newcommand{\SAPSQCPninetyfivetiny}{{\scriptsize{\texttt{\textsc{Saps-QCP-q095}}}}}
\newcommand{\SAPSSWGCPmedtiny}{{\scriptsize{\texttt{\textsc{Saps-SWGCP-med}}}}}
\newcommand{\SAPSSWGCPseventyfivetiny}{{\scriptsize{\texttt{\textsc{Saps-SWGCP-q075}}}}}
\newcommand{\SAPSSWGCPninetyfivetiny}{{\scriptsize{\texttt{\textsc{Saps-SWGCP-q095}}}}}

\newcommand{\SAPSQWH}{{\footnotesize{\texttt{\textsc{Saps-QWH}}}}}
\newcommand{\SAPSQWHtiny}{{\scriptsize{\texttt{\textsc{Saps-QWH}}}}}


\newcommand{\SPEARIBMtwentyfive}{{\footnotesize{\texttt{\textsc{Spear-IBM-q025}}}}}
\newcommand{\SPEARIBMmed}{{\footnotesize{\texttt{\textsc{Spear-IBM-med}}}}}
\newcommand{\SPEARSWVmed}{{\footnotesize{\texttt{\textsc{Spear-SWV-med}}}}}
\newcommand{\SPEARSWVseventyfive}{{\footnotesize{\texttt{\textsc{Spear-SWV-q075}}}}}
\newcommand{\SPEARSWVninetyfive}{{\footnotesize{\texttt{\textsc{Spear-SWV-q095}}}}}

\newcommand{\SPEARIBMtwentyfivetiny}{{\scriptsize{\texttt{\textsc{Spear-IBM-q025}}}}}
\newcommand{\SPEARIBMmedtiny}{{\scriptsize{\texttt{\textsc{Spear-IBM-med}}}}}
\newcommand{\SPEARSWVmedtiny}{{\scriptsize{\texttt{\textsc{Spear-SWV-med}}}}}
\newcommand{\SPEARSWVseventyfivetiny}{{\scriptsize{\texttt{\textsc{Spear-SWV-q075}}}}}
\newcommand{\SPEARSWVninetyfivetiny}{{\scriptsize{\texttt{\textsc{Spear-SWV-q095}}}}}



%%%%%%%%%%%%%%%%%%%%%%%%%%%%%%%%%%%%%%%%%%%%%%
% ABBREVIATIONS
%%%%%%%%%%%%%%%%%%%%%%%%%%%%%%%%%%%%%%%%%%%%%%
\newcommand{\etal}[0]{et al.{}}
\newcommand{\eg}[0]{\emph{e.{}g.{}}}
\newcommand{\ie}[0]{\emph{i.{}e.{}}}
\newcommand{\adhoc}[0]{\emph{ad hoc}}
\newcommand{\cf}[0]{cf.{}}
\newcommand{\vs}[0]{\emph{vs}}
\newcommand{\wrt}[0]{w.{}r.{}t.{}}
\newcommand{\wrtl}[0]{with respect to}
\newcommand{\iid}[0]{i.{}i.{}d.{}}
\newcommand{\naive}[0]{na\"ive}
\newcommand{\Naive}[0]{Na\"ive}

%Math
\newcommand{\vTheta}{{\bm{\Theta}}}
\newcommand{\vtheta}{{\bm{\theta}}}
\newcommand{\vo}{{\bm{o}}}
\newcommand{\calM}{\mbox{${\cal M}$}}
\newcommand{\calD}{\mbox{${\cal D}$}}
\newcommand{\gauss}{\mbox{${\cal N}$}}
\newcommand\transpose{{\textrm{\tiny{\sf{T}}}}}

%% keep figures from going onto a page by themselves
\renewcommand{\topfraction}{0.9}
\renewcommand{\textfraction}{0.1}
\renewcommand{\floatpagefraction}{0.85}


%%%%%%%%%%%%%%%%%%%%%%%%%%%%%%%%%%%%%%%%%%%%%%
% SET UP CHANGEBAR
%%%%%%%%%%%%%%%%%%%%%%%%%%%%%%%%%%%%%%%%%%%%%%

\usepackage[outerbars,color]{changebar}
%\ifx\pdfoutput\undefined
%\else\ifnum\pdfoutput>0
%\usepackage{pdfcolmk}
%\fi\fi

\setcounter{changebargrey}{60}
\cbcolor{red}

\newcommand{\cb}[1]{\cbstart{#1} \cbend}
%\renewcommand{\cb}[1]{#1}
\newcommand{\cbgreen}[1]{\cbstart{\textcolor{green}{#1}} \cbend}
\newcommand{\tred}[1]{\textcolor{red}{#1}}


\newcommand{\denselist}{\itemsep -1.5pt\partopsep -20pt}



%%%%%%%%%%%%%%%%%%%%%%%%%%%%%%%%%%%%%%%%%%%%%%
% CONVENIENT COMMANDS for formulae
%%%%%%%%%%%%%%%%%%%%%%%%%%%%%%%%%%%%%%%%%%%%%%

\newcommand{\nangbra}[1]{$\langle{}$#1$\rangle$}
\newcommand{\angbra}[1]{\langle{}#1\rangle}
\newcommand{\bra}[1]{\left(#1\right)}
\newcommand{\squbra}[1]{\left[#1\right]}
\newcommand{\vctm}[1]{\boldmath{$#1$}\unboldmath{}}
\newcommand{\vct}[1]{\mbox{\boldmath{$#1$}\unboldmath{}}}
%\newcommand{\qed}{~\vspace*{-0.5cm}\hspace*{\textwidth}\qedsymbol~\vspace*{0.5cm}}


%%%%%%%%%%%%%%%%%%%%%%%%%%%%%%%%%%%%%%%%%%%%%%
% COMMANDS for algorithm2e.
%%%%%%%%%%%%%%%%%%%%%%%%%%%%%%%%%%%%%%%%%%%%%%
\SetCommentSty{textit}
\SetKwComment{fhcomment}{// ===== }{}
\newcommand{\nlcom}[1]{{\BlankLine\footnotesize{\fhcomment{\CommentSty{#1}}}}}
\newcommand{\com}[1]{{\footnotesize{\CommentSty{// #1}}}}

\newcommand{\fhem}[1]{\text{\emph{#1}}}
\newcommand{\algo}[2]{#1\;\small #2}

%\newcommand{\Input}[1]{Input: #1}
%\newcommand{\Output}[1]{Output: #1}
%\newcommand{\Effect}[1]{Effect: #1}

\SetKwBlock{Procedure}{begin}{end}
\SetKwFunction{Term}{TerminationCriterion()}
\SetKwFunction{Init}{GenerateInitialSolution}
\SetKwFunction{Acc}{AcceptanceCriterion}
\SetKwFunction{Best}{best}
\SetKwFunction{IteratedLocalsearch}{IteratedLocalsearch}
\SetKwFunction{Localsearch}{LocalSearch}
\SetKwFunction{IterativeFirstImprovement}{IterativeFirstImprovement}
\SetKwFunction{Pert}{Perturbation}
%\DontPrintSemicolon
\SetFuncSty{textit}

%%%%%%%%%%%%%%%%%%%%%%%%%%%%%%%%%%%%%%%%%%%%%%
% COMMANDS for theorems, lemmas, definitions, etc.
%%%%%%%%%%%%%%%%%%%%%%%%%%%%%%%%%%%%%%%%%%%%%%

%\usepackage{amsmath, amsthm, amssymb}
\newtheorem{thm}{Theorem}%[section]
\newtheorem{ex}{Example}%[section]
\newtheorem{lem}[thm]{Lemma}
\newtheorem{cor}[thm]{Corollary}
\newtheorem{obs}[thm]{Observation}

%\theoremstyle{definition}
\newtheorem{define}[thm]{Definition}
\hyphenation{ge-ne-ral-ize}

%%%%%%%%%%%%%%%%%%%%%%%%%%%%%%%%%%%%%%%%%%%%%%
% FIGURES
%%%%%%%%%%%%%%%%%%%%%%%%%%%%%%%%%%%%%%%%%%%%%%

\newcommand{\largequadraticgraph}[1]{
%        \includegraphics[width=3.9cm,height=3.9cm]{#1}
%        \includegraphics[width=4.6cm,height=4.6cm]{#1}
%        \includegraphics[width=4.9cm,height=4.6cm]{#1}
%        \includegraphics[width=5.4cm,height=5.4cm]{#1}
        \includegraphics[width=5.4cm,height=5.4cm]{#1}
  }


\newcommand{\smallquadraticgraph}[1]{
%        \includegraphics[width=3.9cm,height=3.9cm]{#1}
%        \includegraphics[width=4.6cm,height=4.6cm]{#1}
        \includegraphics[width=4.9cm,height=4.6cm]{#1}
%        \includegraphics[width=5.4cm,height=5.4cm]{#1}
}


\newcommand{\quadraticgraph}[1]{
%        \includegraphics[width=4.1cm,height=4.1cm]{#1}
        %\includegraphics[width=4.6cm,height=4.6cm]{#1}
        \includegraphics[width=5.4cm,height=5.4cm]{#1}
        %\includegraphics[width=6.15cm,height=6.15cm]{#1}
}

\newcommand{\smallgraph}[1]{
%        \includegraphics[width=6.15cm,height=4.1cm]{#1}
        %\includegraphics[width=6.9cm,height=4.6cm]{#1}
%        \includegraphics[width=12cm,height=8cm]{#1}
\includegraphics[width=5.4cm,height=3.6cm]{#1}
%	\includegraphics[width=4.95cm,height=3.3cm]{#1}
}


\newcommand{\normalgraph}[1]{
%        \includegraphics[width=6.15cm,height=4.1cm]{#1}
        \includegraphics[width=6.9cm,height=4.6cm]{#1}
%        \includegraphics[width=12cm,height=8cm]{#1}
%\includegraphics[width=5.4cm,height=3.6cm]{#1}
%	\includegraphics[width=4.95cm,height=3.3cm]{#1}
}

\newcommand{\biggergraph}[1]{
%        \includegraphics[width=6.15cm,height=4.1cm]{#1}
        \includegraphics[width=6.9cm,height=4.6cm]{#1}
%        \includegraphics[width=12cm,height=8cm]{#1}
%\includegraphics[width=5.4cm,height=3.6cm]{#1}
%	\includegraphics[width=4.95cm,height=3.3cm]{#1}
}

\newcommand{\fig}[3]{  % input_file, label, caption
  \begin{figure}[tpb]
    \begin{center}
      \normalgraph{#1}
      \caption{\label{#2}#3}
    \end{center}
  \end{figure}
}


\newcommand{\qfig}[3]{  % input_file, label, caption
  \begin{figure}[tpb]
%hh: temporarily changed:
%  \begin{figure}[p]
    \begin{center}
      \quadraticgraph{#1}
      \caption{\label{#2}#3}
    \end{center}
  \end{figure}
}

\newcommand{\largeqfig}[3]{  % input_file, label, caption
%  \begin{figure}[tpb]
%hh: temporarily changed:
  \begin{figure}[p]
    \begin{center}
      \largequadraticgraph{#1}
      \caption{\label{#2}#3}
    \end{center}
  \end{figure}
}

\newcommand{\sides}[6]{         %fig1, fig2, cap1, cap2, cap, label
        \begin{figure*}[tbph]
        \hfill
            \subfigure[#3]
            {
                \quadraticgraph{#1}
                  }
        \hfill{}
        \hfill
            \subfigure[#4]
            {
                \quadraticgraph{#2}
                  }
        \hfill{}
        \caption{#5}
        \label{#6}
%        \vspace{-0.245cm}
        \end{figure*}
}

\newcommand{\threefig}[8]{         %fig1, fig2, fig3, cap1, cap2, cap3, cap, label
    \begin{figure*}[tbp]
      \begin{center}
        \mbox{
            \subfigure[#4]
            {
                \quadraticgraph{#1}
                  }
            \subfigure[#5]
            {
                \quadraticgraph{#2}
                  }
            \subfigure[#6]
            {
                \quadraticgraph{#3}
                  }
        }
        \vspace{-0.4cm}
        \caption{#7}
        \label{#8}
      \end{center}
      \vspace{-0.245cm}
    \end{figure*}
}

\newcommand{\threebfig}[8]{         %fig1, fig2, fig3, cap1, cap2, cap3, cap, label
    \begin{figure*}[tb]
      \begin{center}
        \mbox{
            \subfigure[#4]
            {
                \smallgraph{#1}
                  }
            \subfigure[#5]
            {
                \smallgraph{#2}
                  }
            \subfigure[#6]
            {
                \smallgraph{#3}
                  }
        }
        \vspace{-0.4cm}
        \caption{#7}
        \label{#8}
      \end{center}
      \vspace{-0.245cm}
    \end{figure*}
}

\usepackage{fullpage}
\usepackage{setspace}
\usepackage{subfiles}
\usepackage{graphicx}
\usepackage{color}
%%\usepackage{hyperref}
\usepackage{enumitem}
\usepackage{probsoln}
\usepackage[tikz]{bclogo}
\PSNrandseed{\time}
\newcommand{\note}[1]{}
% comment the next line to turn off notes
\renewcommand{\note}[1]{~\\\frame{\begin{minipage}[c]{\textwidth}\vspace{2pt}\center{#1}\vspace{2pt}\end{minipage}}\vspace{3pt}\\}


\begin{document}

\title{Manual for SMAC version vDEV}

\author{
Frank~Hutter \& Steve~Ramage\\
Department of Computer Science\\
University of British Columbia\\
Vancouver, BC\ \ V6T~1Z4, Canada\\
\texttt{\{hutter,seramage\}@cs.ubc.ca}
}


\maketitle

\tableofcontents

\subfile{manual-intro}
\section{Differences Between SMAC and ParamILS}

There are a number of differences between SMAC and ParamILS, including the following.

\begin{itemize}
\item \textbf{Support for continuous parameters:} 
While ParamILS was limited to categorical parameters, SMAC also natively handles continuous and integer parameters. See Section \ref{sec:param_decl_clauses} for details.

\item \textbf{Run objectives:} 
Not all of ParamILS's run objectives are supported at this time. If you require an unsupported objective please let us know.

\item \textbf{Order of instances:} 
In contrast to ParamILS, the order of instances in the instance file does not matter in SMAC.

\item \textbf{Configuration time budget and runtime overheads:} 
Both ParamILS and SMAC accept a time budget as an input parameter. ParamILS only keeps track of the CPU time the target algorithm reports and terminates once the sum of these runtimes exceeds the time budget; it does \emph{not} take into account overheads due to e.g.\ command line calls of the target algorithm. In cases where the reported CPU time of each target algorithm run was very small (e.g.\ milliseconds), these unaccounted overheads could actually dominate ParamILS's wall-clock time.
SMAC offers a more flexible management of its runtime overheads through the options 
\textbf{-~$\!$-use-cpu-time-in-tunertime} and \textbf{-~$\!$-wallclock-limit}. See Section \ref{sec:wall-clock} for details on the wall clock time limit.

\item \textbf{Resuming previous runs:} 
While this was not possible in ParamILS, in SMAC you can resume previous runs from a saved state.
Please refer to Section \ref{sec:state-restoration} for how to use the state restoration feature. Section \ref{subsec:state-files} describes the file format for saved states.

\item \textbf{Feature files:} 
SMAC accepts as an optional input a feature file providing additional information about the instances in the training set; see Section \ref{sec:feature_file_format}.


\item \textbf{Algorithm wrappers:} 
The wrapper syntax has been extended in SMAC to support additional results in the ``solved'' field. Specifically, there is a new result \textbf{ABORT} signalling that the configuration process should be aborted (e.g. because the wrapper is in an inconsistent state that should never be reached). A similar behaviour is triggered if option \textbf{-~$\!$-abort-on-first-run-crash} is set and the first run returns \textbf{CRASHED}. Additionally, the wrapper can also return additional data to SMAC that is associated with the run \footnote{This data will be saved in the run and results file (Section \ref{subsec:state-files}) that is used in state saving}. For more information see Section \ref{sec:wrapper_output}.

\item \textbf{Instance files vs. instance/seed files:} 
The \textbf{instance\_file} parameter now auto-detects whether the file conforms to ParamILS's \textbf{instance\_file} or \textbf{instance\_seed\_file} format. SMAC treats the latter option as an alias for the former. See Section \ref{sec:instance_file_format} for details.
While SMAC is backwards compatible with previous (space-separated) files, the preferred format is now \texttt{.csv}.
\end{itemize}

  
\section{Commonly Used Options}
\subfile{commonly-used-options}

%%%%%%%%%%%%%%%%%%%%%%%%%%%%%%%%%%%%%%%%%%% 

\section{File Format Reference} 
%%%%%%%%%%%%%%%%%%%%%%%%%%%%%%%%%%%%%%%%%%%
\subfile{file-formats}


\section{Wrappers}
\subfile{wrapper}


%%%%%%%%%%%%%%%%%%%%%%%%%%%%%%%%%%%%%%%%%%%%%%%%%%%%%%%%%%%%%%%%%%%%
\section{Interpreting SMAC's Output}
%%%%%%%%%%%%%%%%%%%%%%%%%%%%%%%%%%%%%%%%%%%%%%%%%%%%%%%%%%%%%%%%%%%%


\subfile{output}
%%\documentclass[manual.tex]{subfiles} 
\begin{document}

%%%%%%%%%%%%%%%%%%%%%%%%%%%%%%%%%%%%%%%%%%%%%%%%%%%%%%%%%%%%%%%%%%%%
\subsection{Option Files}
%%%%%%%%%%%%%%%%%%%%%%%%%%%%%%%%%%%%%%%%%%%%%%%%%%%%%%%%%%%%%%%%%%%%
Option Files are a way of saving a different set of values 
frequently used with SMAC without having to specify them on every execution. 
The general format for an option file is the name
of the configuration option (without the two dashes), an equal sign, and then the value (for booleans it should be true or false, lowercase). Currently options that take multiple arguments are not supported. Additionally you can not use aliases that are single dashed (\eg{ to override the Experiment Directory, you must use \textbf{-~$\!\!$-experiment-dir} and not \textbf{-e}})  

When using Option Files it is important that no two files (including the Scenario File), specify the same option, the resulting configuration is undefined, and in general this will not throw an error.

%%%%%%%%%%%%%%%%%%%%%%%%%%%%%%%%%%%%%%%%%%%%%%%%%%%%%%%%%%%%%%%%%%%%
\subsubsection{Scenario File}
%%%%%%%%%%%%%%%%%%%%%%%%%%%%%%%%%%%%%%%%%%%%%%%%%%%%%%%%%%%%%%%%%%%%

The Scenario Option File, or Scenario File, is the recommended way of configuring SMAC \footnote{Nothing in general prevents you from specifying non-scenario options in these files, but in general you should restrict your files to these.}.
The Scenario Files used in SMAC are backwards compatible with ParamILS and the name of option names here reflect that\footnote{Every option name listed here is in fact an alias for an existing option listed in the section \ref{sec:options-ref} and it is entirely possible to use SMAC without using Scenario Files.}. \textsc{Note:} \textbf{cutoff\_length} is not currently supported.

\begin{description}
\item [algo] An algorithm executable or wrapper script around
an algorithm that conforms with the input/output format specified
in section \ref{sec:exec-spec}. The string here should be invokable via the system shell.
\item [{execdir}] Directory to execute \texttt{<algo>} from: (\ie{ ``cd
\texttt{<execdir>}; \texttt{<algo>}'' })
\item [{deterministic}] A boolean that governs whether or not the algorithm should be treated as deterministic. For backwards compatibility with ParamILS, this option also supports using 0 for false, and 1 for true. SMAC will never invoke the target algorithm more than once for any given instance, seed and configuration. If this is set to \texttt{true}, SMAC will never invoke the target algorithm more than once for any given instance and configuration.

\item [{run\_obj}] Determines how to convert the resulting output line (as defined in Section
 \ref{sec:wrapper_output}) into a scalar quantifying how ``good'' a single algorithm execution
is, (\eg{ how long it took to execute, how good of a solution it found,
etc...}). SMAC will attempt to \emph{minimize} this objective. 

 Currently implemented objectives are the following:
\item [{
\begin{tabular}{|c|c|}
\hline 
Name & Description\tabularnewline
\hline 
\hline 
RUNTIME & Minimize the reported runtime of the algorithm.\tabularnewline
\hline 
QUALITY & Minimize the reported quality of the algorithm.\tabularnewline
\hline 
\end{tabular}}]~
\item [{overall\_obj}] While \textbf{run\_obj} defines the objective function
for a single algorithm run, \textbf{overall\_obj} defines how those
single objectives are combined to reach a single scalar value to compare
two parameter configurations. Implemented examples for this are as
follows:
\item [{%
\begin{tabular}{|c|c|}
\hline 
Name & Description\tabularnewline
\hline 
\hline 
MEAN & The mean of the values\tabularnewline
\hline
MEAN1000 & Unsuccessful runs are counted as 1000 $\times$ \textbf{target\_run\_cputime\_limit}\tabularnewline
\hline 
MEAN10 & Unsuccessful runs are counted as 10 $\times$ \textbf{target\_run\_cputime\_limit}\tabularnewline
\hline 
\hline 
\end{tabular}}]~
\item [{target\_run\_cputime\_limit}] The CPU time after which a single algorithm execution
will be terminated as unsuccess (and treated as a \textbf{TIMEOUT}). This is an important parameter: If chosen too high, lots of time will be wasted with unsuccessful
runs. If chosen too low the optimization is biased to perform well
on easy instances only. 
%\item [{cutoff\_length}] The runlength after which a single algorithm execution
%will be terminated unsuccesfully. The actual semantic meaning of this
%value is up to the target algorithm. NOT SUPPORTED IN THIS VERSION)
\item [{cputime\_limit}] The limit of the CPU time allowed for configuration (\ie{The sum of all algorithm runtimes, and by default the sum of the CPU time of SMAC itself}).
\item [{wallclock\_limit}] The limit of the amount of wallclock (or real) time allowed for configuration. 
\item [{paramfile}] Specifies the file with the parameters of the algorithm.
The format of this file is covered in Section \ref{sec:paramfile}.
\item [{outdir}] Specifies the directory SMAC should write its results
to. 
\item [{instance\_file}] Specifies the file containing the list of problem instances (and possibly seeds) for SMAC to use during the \emph{Automatic Configuration Phase}. The ParamILS parameter \textbf{instance\_seed\_file} aliases this one and the format is auto-detected. The format of these files is covered in section \ref{sec:instance_file_format}.
\item [{test\_instance\_file}] Specifies the file containing the list of problem instances (and possibly seeds) for SMAC to use during \emph{Validation Phase}. The ParamILS parameter \textbf{test\_instance\_seed\_file} aliases this one and the format is auto-detected. The format of these files is covered in section \ref{sec:instance_file_format}.
\item [{feature\_file}] Specifies the a file with the features for the instances in the \textbf{instance\_file} and possibly the \textbf{test\_instance\_file} \footnote{The Validator will load features into memory for test instances if they exist.}. The format of this file is covered in section \ref{sec:feature_file_format}.

\end{description}

%%%%%%%%%%%%%%%%%%%%%%%%%%%%%%%%%%%%%%%%%%%%%%%%%%%%%%%%%%%%%%%%%%%%
\subsection{Instance File Format} \label{sec:instance_file_format}
%%%%%%%%%%%%%%%%%%%%%%%%%%%%%%%%%%%%%%%%%%%%%%%%%%%%%%%%%%%%%%%%%%%%


The files used by the \textbf{instance\_file} \& \textbf{test\_instance\_file} options 
come in four potential formats, all of which are CSV based\footnote{Specifically each cell should be double-quoted (\ie{''}), and use a comma as a cell delimiter. SMAC also supports the old method of reading files that use space as a cell delimiter and do not enclose values. However these files cannot handle \textbf{Instance Name}'s that contain spaces.}. Before specifying the formats it is important to note the three kinds of information that are specified with instances \footnote{Features which are required for SMAC but not ParamILS are specified in a seperate file see section \ref{sec:feature_file_format}.}.

\begin{description}
\item[Instance Name] The name of the instance that was selected. This should be meaningful to the target algorithm we are configuring \footnote{Generally \textbf{Instance Names} reference specific files on disk.}.
\item[Instance Specific Information] A free form text string (with no spaces or line breaks) that will be passed to the Target Algorithm whenever executed.
\item[Seed] A specific seed to use when executing the target algorithm.
\end{description}

The possible formats are as follows, and depend on what information you'd like to specify.

\begin{enumerate}
\item	Each line specifies only a unique \textbf{Instance Name}. No \textbf{Instance Specific Information} will be used, and \textbf{Seed}'s will be automatically generated.

\item  Each line specifies a \textbf{Seed} followed by the \textbf{Instance Name}. Every line must be unique, but for each \textbf{Instance Name} additional seeds will be used in order, when that instance is selected.

\item Each line specifies a \textbf{Instance Name} followed by the \textbf{Instance Specific Information}. Every \textbf{Instance Name} must be unique, \textbf{Seed}'s will be automatically generated.

\item Each line specifies a \textbf{Seed} followed by the \textbf{Instance Name} followed by the \textbf{Instance Specific Information}. Every line must be unique, and furthermore, for all \textbf{Instance Name}'s the \textbf{Instance Specific Information} must be the same for all \textbf{Seed} values (\ie{You cannot specify different instance specific information that is a function of the seed used}).

\end{enumerate}

%%%%%%%%%%%%%%%%%%%%%%%%%%%%%%%%%%%%%%%%%%%%%%%%%%%%%%%%%%%%%%%%%%%%
\subsection{Feature File Format}\label{sec:feature_file_format}
%%%%%%%%%%%%%%%%%%%%%%%%%%%%%%%%%%%%%%%%%%%%%%%%%%%%%%%%%%%%%%%%%%%%



The \textbf{feature\_file} specifies features that are to be used for instances. Feature Files are specified in CSV format, the first column of every row should list an \textbf{Instance Name} as it appears in the \textbf{instance\_file}. The subsequent columns should list \texttt{double} values specifying a computed continuous feature. By convention the value $-512$, and $-1024$ are used to signify that a feature value is missing or not applicable. All instances must have the same number of features.

At the top of the file there \textsc{must} appear a header row, the cell that appears above the instance names is unimportant, but for each feature a unique and \emph{non-numeric} (\ie{ it must contain atleast one letter}) feature name must be specified.

%%%%%%%%%%%%%%%%%%%%%%%%%%%%%%%%%%%%%%%%%%%%%%%%%%%%%%%%%%%%%%%%%%%%
\subsection{Parameter Configuration Space Format} \label{sec:paramfile}
%%%%%%%%%%%%%%%%%%%%%%%%%%%%%%%%%%%%%%%%%%%%%%%%%%%%%%%%%%%%%%%%%%%%

 \subfile{pcs-subfile}


\end{document}


 
\section{Developer Reference}

\subfile{dev-ref}

%%%%%%%%%%%%%%%%%%%%%%%%%%%%%%%%%%%%%%%%%%%%%%%%%%%%%%%%%%%%%%%%%%%%
\section{Acknowledgements}
%%%%%%%%%%%%%%%%%%%%%%%%%%%%%%%%%%%%%%%%%%%%%%%%%%%%%%%%%%%%%%%%%%%%

We are indebted to Jonathan Shen for porting our random forest code from C to Java in preparation for a Java port of all of SMAC. Alexandre Fr\'echette and Chris Thornton for their constant feedback and patches to SMAC.  We would also like to thank Marius Lindauer for many valuable early bug reports and suggestions for improvements, as well as subsequent patches.

%%%%
% Steve Ramage flipped a coin on August 3rd, witnessed by Alex Fr\'echette who called Tails, and it landed tails so he goes first
%%%


\renewcommand{\bibsection}{\section{References}}
\bibliographystyle{apalike}
\bibliography{short,frankbib}


\section{Appendix}
\subsection{Return Codes}

\begin{table}[h]
\begin{tabular}{| c | c | c |}
\hline
Value & Error Name & Description \\
\hline
\hline
0 & \textbf{Success} & Everything completed successfully \\
\hline
1 & \textbf{Parameter Error} & There was a problem with the input arguments or files  \\
\hline
2 & \textbf{Trajectory Divergence} & For some reason SMAC has taken a unexpected path \\

& & (\eg{} SMAC executes a run that does not match a run \\
& & in the \textbf{-~$\!\!$-runHashCodeFile}) \\
\hline
3 & \textbf{Serialization Exception} & A problem occurred when saving or restoring state \\
\hline
255 & \textbf{Other Exceptions} & Some other error occurred \\
\hline
\end{tabular}
\end{table}

\textsc{NOTE:} All error conditions besides 255 are fixed. However in future some exceptions that previously reported 255 may be changed to a non 255 value as needed / requested
\\
\\
\\
\\
\\
\\
\subsection{Version History of Java SMAC}
	\begin{description}
		\item[Version 2.00 (Aug-2012)] First Internal Release of Java SMAC (this is a port and extension of the original Matlab version).
		\item[Version 2.02 (Oct-2012)] First Public Release of SMAC v2 and contained many fixes from the previous release.
		\item[Version 2.04 (Dec-2012)] Second Release of Java SMAC including the following improvements:
			\begin{enumerate}
			\item Validation file output times consistent with Tuner Times
			\item Some \textbf{INFO} log statements have been moved to \textbf{DEBUG} and some \textbf{DEBUG} to \textbf{TRACE}
			\item Added support for verifying whether responses of SAT and UNSAT are consistent with Instance Specific Information see \textbf{-$~\!$-verifySAT} option for more information
			\item Added support for the SMAC\_MEMORY environment variable to control how much RAM (in MB) SMAC will use when  executed via the supplied shell scripts. 
			\item Context is now added to the state folders to make it easier to debug issues later, to disable consult the \textbf{-$~\!$-saveContext} option.
			\item Greatly improved memory usage in State Serialization code, and now we free the existing model prior to building a new one, so for some JVMs this may improve memory usage.
			\end{enumerate}

			\item[Version 2.04.01 (Feb-2013)] Minor Bug Fix of Java SMAC
				\begin{enumerate}
 			\item Added option to validate over training set instances 
			\item Can now use $<$DEFAULT$>$ as a configuration to validate against
			\item Fixed bug where \textbf{TIMEOUT} runs below our requested cutoff time are not counted properly when considering incumbent changes
			\item Can now specify the initial incumbent with the \textbf{-$~\!$-initialIncumbent} option.
			\item Wallclock time is now saved in the trajectory file instead of -1
			\item FAQ Improvements
			\item Git commit hash is now documented in Manual, FAQ, and Version strings
			\item \textbf{(BETA)} Support for bash auto-completion of arguments for \texttt{smac} and \texttt{smac-validate}. You can load the file by running: 
\begin{verbatim}
. ./util/bash_autocomplete.sh
\end{verbatim}

			\end{enumerate}

			 \item[Version 2.04.02 (Aug-2013)] Minor Bug Fix of Java SMAC
	              \begin{enumerate}
                     \item Incumbent Performance now displayed when validation is turned off.
                     \item \textbf{-$~\!$-runtimeLimit} option is no longer just for show.
                  \end{enumerate}

			 \item[Version 2.06.00 (Aug-2013)] Significant Feature Enhancements
	              \begin{enumerate}
	              	 \item New \texttt{algo-test} utility allows easy invocation of wrappers.
	              	 \item New \texttt{verify-scenario} utility preforms extra validation on scenario files.
                     \item Scenario now ends if the configuration space is exhausted
					 \item SMAC now lets you search only a subspace for good configurations
					 \item Validation output formats improved with headers
					 \item Option to always compare with the initial incumbent (to prevent an early poor choice from derailing the run) (See \textbf{-$~\!$-always-run-initial-config})
					 \item SMAC reports an error if runs give different answers for \textbf{SAT} and \textbf{UNSAT} now
					 \item New \textbf{-$~\!$-restore-scenario} option to make restoring scenarios easier
					 \item New \textbf{-$~\!$-warmstart} option makes it possible to preload the model with additional SMAC runs.
					 \item Can now set seeds to different parts of SMAC using \textbf{-S}
					 \item Runtime Statistics and Termination Reasons now rewritten
					 \item New validation options \textbf{-$~\!$-validate-all}, \textbf{-$~\!$-validate-only-if-tunertime-reached} (See the validation options for all of them)
					 \item SMAC now checks limits \textit{before} scheduling a run, rather than immediately after the run as in previous versions. (This means that if the last run went over, but changed the incumbent it will be logged.)
					 \item Instances can now be ordered deterministically (that is in the order they are declared in the instance file via \textbf{-$~\!$-determinstic-instance-ordering}.
					 \item Usage improved via new help levels which are displayed with \textbf{-$~\!$-help-level} and new usage screens.
					 \item Improvements to bash auto completion.
					 \item Target Algorithm Evaluators now take options.
					 \item Fixes for CPU Time calculation in SMAC.
					 \item Example scenarios cleaned up, new ones provided.
					 \item SMAC should be more forgiving with relative paths in a scenario file.
					 \item Default option files now supported (SMAC will read from \texttt{\textasciitilde/.aclib/smac.opt}, \texttt{\textasciitilde/.aclib/tae.opt} and \texttt{\textasciitilde/.aclib/help.opt}. It will also read from defaults for plugins that are available. \textbf{NOTE:} A future version changed the files to \texttt{\textasciitilde/.aeatk/}.
					 \item Rungroup name is now configurable.
					 \item Logging of some objects is cleaned up.
					 \item Windows Startup scripts, and improved Unix start up scripts.
					 \item Fixed lock-up issue with wrappers launching unterminating subprocesses.
					 \item Fixed ConvergenceException error message.
					 \item Options now have a primary non-camel case format.
					 \item Manual now has a basic options section, before listing all the options.
					 \item Significant API changes to the Target Algorithm Evaluators so previous plugins will need to be refactored (and another change will come either in v2.06 or v2.08).
					 \item SMAC will now match capitalization of words in the Result String of wrappers.
					 \item New \textbf{-$~\!$-validation-seed} option should cause the validation at the end of SMAC to behave the same as the stand-alone utility.
                  \end{enumerate}

			\item[Version 2.06.01 (Oct-2013)] Minor Bug Fix Release of Java SMAC
	              \begin{enumerate}
	              	 \item Fixed a bug introduced in 2.06.00 that caused validation to be performed against the training instance distribution instead of the test instance distribution.
                     \item Default acquisition function for solution quality optimization is now Expected Improvement (instead of Exponential Expected Improvement).
                     \item Fixed exception if Scenario file doesn't have extension.
                     \item New option \textbf{-$~\!$-terminate-on-delete} will cause SMAC to abort the procedure before the next set of runs (as if it had hit it's CPU time limit) if the file specified is deleted.
                     \item New option \textbf{-$~\!$-kill-runs-on-file-delete} will cause SMAC to kill any runs in progress . This option should be used with care, as it may cause SMAC to select the wrong incumbent, and it should always be used with \textbf{-$~\!$-terminate-on-delete}.
                     \item New option \textbf{-$~\!$-save-runs-every-iteration} will cause SMAC to output the runs and results file necessary to restore state every run. This is useful if your cluster or environment is particularly unreliable. It should \textsc{NOT} be used when runtimes in the scenario can grow very small as the amount of time SMAC will spend writing to disk loosely \footnote{Assuming the number of iterations scales linearly with the number of runs.} changes from $O(n)$ to $O(n^2)$, where $n$ is the number of runs it performs.
                     \item If SMAC is shutting down for an unexpected reason (e.g. \texttt{OutOfMemoryError} ,or it received a \texttt{SIGTERM}), SMAC will now try its best to write a final batch of state data with the "SHUTDOWN" prefix. \\
                      \textsc{Note:} This state may be corrupted for a variety of reasons, and even if it is written correctly you may not be able to restore it properly as the snap shot may be from the middle of an iteration.
                     \item Fixed typo in error message that mistakenly reported that instances where missing, when in fact it was the test instances that were missing.                     
                  \end{enumerate}
                  
          \item[Version 2.08.00 (Aug-2014)] Usability and Validation Changes
          	\begin{enumerate}
          		\item SMAC is now more picky about instance names and feature names matching.
          		\item New \texttt{sat-check} utility allows determination of the satisfiability for each instance of a instance file.
          		\item Environment variable \textbf{AEATK\_CONCURRENT\_TASK\_ID} is now set when executing the wrapper, containing an index into the number of concurrent jobs. This is primarily used to allow the wrapper to determine CPU affinities correctly. See the wrapper section for more information.-
          		\item SMAC has been made drastically less verbose. The default level \texttt{INFO} now only contains the final information, and information about changes to the incumbent. \texttt{DEBUG} contains most of the old info level, \texttt{TRACE} contains most of the old \texttt{DEBUG} levels. The old \texttt{TRACE} level was never used and has been removed.
          		\item Instances can now be specified by folder using the \textbf{-$~\!$-instances} and \textbf{-$~\!$-test-instances} option. You can restrict which instances are used via the \textbf{-$~\!$-instance-suffix}  and \textbf{-$~\!$-test-instance-suffix} 
          		
          		\item \textbf{-$~\!$-exec-dir} option now defaults to current working directory.
          		\item New option \textbf{-$~\!$-use-instances} will use a dummy instance instead of the instance file (useful for black box optimization).
          		\item New advanced option \textbf{-$~\!$-shared-model-mode} may improve performance in some cases, see Section \ref{subsec:sharedModelMode}.
          		\item \textbf{[BETA]} Target Algorithm Evaluator implementation allows integrating with a TAE using UDP/TCP (more to come).
          		\item Implemented a work around to a bug where configurations with censored early runs could become the incumbent erroneously. It's still suboptimal, but in fact it probably would never happen. See Known Issue \#1 in section \ref{known-issues}.
          		\item Validation rounding mode now changes the number of runs on deterministic runs, or runs with set problem instance seed pairs.
          		\item New option \textbf{-$~\!$-cli-kill-by-environment-cmd} allows terminating all processes by an environment variable. See section \ref{sec:exec-env} for more information.
          		\item Target algorithms no longer see quoted arguments for parameter values. The option \textbf{-$~\!$-cli-call-params-with-quotes} can be used to get the old behaviour back, this option will likely be removed in future.
          		\item New option \textbf{-$~\!$-quick-saves} controls whether to make any quick save states or not.
          		\item New option \textbf{-$~\!$-intermediary-saves} controls whether to many any save states at all while SMAC is running (if false SMAC will still save information at the end)
          		\item Revamped Quickstart guide
          		\item After validation SMAC prints correct termination reason message.
          		\item SMAC will now terminate all outstanding runs when exiting prematurely (for instance due to \textbf{CTRL+C})
          		\item Standardized scenario options (no new ones), but scenario options can be used in ParamILS versions 2.3.7 and later.
          		
          		\item Mitigated bug that caused deterministic instances to take forever to load from file. This may still happen in some cases, if feature file names and instance file names do not perfectly match up.
          		\item Auto detect restore scenario option now made more robust in case files are missing
          		\item The state merge utility no longer crashes if merging runs that don't have a run for every instance
          		\item Renamed many references of ACLib to AEAToolkit to reflect change of name of the toolkit SMAC is built with.
          		\item Default options are now read from \texttt{\textasciitilde/.aeatk} instead of \texttt{\textasciitilde/.aclib}.
				\item Fixed an issue with absolute paths on windows not being handled correctly.
				\item Validation now performs 1 run per instance by default instead of next multiple after 1000.
				\item Can now specify the number of cores that SMAC validate will use (only when using the local command line), using the \textbf{-$~\!$-validation-cores} option.
				\item Previous state folder is now renamed to something that preserves the run name and is no longer a warning.
				\item Emphasized in many places that SMAC is minimizing the objective functions.
				\item SMAC now ignores the seed output in the response of wrappers entirely (it automatically substitutes the requested value. If your wrapper doesn't set this value correctly, you may notice discrepancies in SMAC.
				\item A few validation options have been deprecated and removed
				\item Can now validate multiple trajectory files in one pass using the \textbf{-$~\!$-trajectory-files} option. 
				\item Output format of validation has been completely changed to be more useful.
				\item \texttt{traj-run-N.csv} is now \texttt{detailed-traj-run-N.csv} and has a slightly different format.
				\item SMAC now requires Java version 7 to run.
				\item \texttt{conf/logback.xml} is no longer used, and the file is stored internally. To override the configuration, set the java system property \texttt{logback.configurationFile=/path/to/config.xml}
				\item Some columns in the trajectory file have been renamed for clarity. The order is still the same.
				\item Changed default wrapper string to ``Result of this algorithm run:''.
				
          	\end{enumerate}
		\item[Version 2.10.00 (May-2015)] Feature Improvements
			\begin{enumerate}
				\item In shared model mode, SMAC can now reuse an existing run from another run, instead of re-running it.
				\item Detailed Trajectory File now outputs predicted performance of the incumbent.
				\item Fixed a bug with the LCB acquisition function that (probably) causes poor performance.
				\item Added Mac OS X support for SAPS example scenario (thanks Chris Fawcett and Alexandre Fr\'{e}chette).
				\item Removed explicit SAPS Windows scenario, and changed the Linux one to detect windows.
				\item Fixed a bug with solution quality optimization and large values greater than the cutoff time. \textbf{CRASHED} runs will now be penalized. For more details about how CRASHED runs are handled see Section \ref{subsec:wrapper-output-semantics}.
				\item Shared model mode will now log a message when a new file to read from is detected.
				\item Fixed spear scenario to use instance features again.
				\item SMAC will now throw an error if feature file specified and no features
				found.
				\item Runs that are terminated because of taking too long will now be treated as \textbf{CRASHED} instead of \textbf{TIMEOUT}.
				\item New PCS Syntax (old syntax still works fine) [Thanks to Marius].
				\item Support for ordinals [Thanks to Marius].
				\item More advanced support for conditionals [Thanks to Marius].
				\item New Advanced Forbidden Syntax.
				\item Vastly improved memory usage.
				\item Fixed bug related to integer parameters being rerun.
				\item Can now set \textbf{-$~\!$-num-ei-random} to zero.
				\item Can now specify an initial list of challengers for SMAC using the 
				\textbf{-$~\!$-initial-challengers}. [Thanks to Matthias]
				\item SMAC will no longer call algorithm with zero cut off when using other TAEs.
				\item Fixed \texttt{IllegalStateException} when dealing with large response values.
				\item SMAC no longer leaves temp files on disk.
				\item Fixed walltime reported incorrectly for some \textbf{CRASHED} runs.
				\item New option \textbf{-$~\!$-num-ls-random} allows using random points for starting local search (defaults to zero).
				\item License changed to AGPLv3.
			\end{enumerate}
	\item[Version 2.10.01 (June-2015)] Minor Fix
		\begin{enumerate}
			\item Undeprecated \textbf{-$~\!$-use-scenario-outdir} after an outcry from users.
		\end{enumerate}
	\item[Version 2.10.02 (July-2015)] Minor Fixes
		\begin{enumerate}
			\item Fixed bug with \textbf{-$~\!$-initial-incumbent-runs} not being able to be set higher than the number of instances (as opposed to number of instance and seed pairs.
			\item Fixed bug with \textbf{-$~\!$-initial-challenger-runs} causing a crash in rare circumstances.
			\item Fixed minor bug with pseudo-random number generation (PRNG) which caused random sequences used in the first iteration, to be replayed in the second. The most pronounced effect of this, was that random configurations generated in the second iteration would be identical to those in the first. 
			\item Added default directory for \textbf{-$~\!$-file-cache-source} and \textbf{-$~\!$-file-cache-output} of the folder runcache in the current working directory.
		\end{enumerate}
				
		

	\end{description}
\subsection{Known Issues}
\label{known-issues}
\begin{enumerate}
\item In a rare case, configurations that are reinspected by SMAC after initially being rejected may continue their challenge when they otherwise shouldn't. If the configuration continues it's challenges successfully, prior to being the incumbent we will presently check all the runs, which is strictly more expensive than necessary.

\item Using any alias for \textbf{-$~\!$-showHiddenParameters}, \textbf{-$~\!$-help}, or \textbf{-$~\!$-version} as values to other arguments (\eg{ Setting -$~\!$-runGroupName -$~\!$-help}) does not parse correctly (This is unlikely to be fixed until someone complains).
\item Using large parameter values in continuous integral parameters, may cause loss of precision, and or crashes if the values are too big.

\item ArrayOutOfBoundsException occurs if not all instances have features

\item \textbf{-$~\!$-num-seeds-per-test-instance} and \textbf{-$~\!$-num-test-instances} are both broken currently and will probably be removed in the future.


\end{enumerate}

\clearpage


\subsection{Basic Options Reference}
The following sections outline only the basic options
\label{sec:options-basic-ref}
\subfile{options-basic-ref}


\clearpage

\subsection{Complete Options Reference}
\label{sec:options-ref}
\subfile{options-ref}

\end{document}

