\documentclass[manual.tex]{subfiles} 
\begin{document}


%%%%%%%%%%%%%%%%%%%%%%%%%%%%%%%%%%%%%%%%%%%%%%%%%%%%%%%%%%%%%%%%%%%%
\subsection{Algorithm executable / wrapper}
%%%%%%%%%%%%%%%%%%%%%%%%%%%%%%%%%%%%%%%%%%%%%%%%%%%%%%%%%%%%%%%%%%%%

\label{sec:exec-spec} The target algorithm as specified by the \textbf{algo} parameter must obey the following general contracts. While modifying your own code to directly achieve this is one option there are other methods outlined in section \ref{sec:exec-options}.

%%%%%%%%%%%%%%%%%%%%%%%%%%%%%%%%%%%%%%%%%%%%%%%%%%%%%%%%%%%%%%%%%%%%
\subsubsection{Invocation}
%%%%%%%%%%%%%%%%%%%%%%%%%%%%%%%%%%%%%%%%%%%%%%%%%%%%%%%%%%%%%%%%%%%%

The algorithm must be invokable via the system command-line using the following command with arguments:

\texttt{<algo\_executable> <instance\_name> <instance\_specific\_information>
<cutoff\_time> <cutoff\_length> <seed> <param>} \texttt{<param>} \texttt{<param>}...

\begin{description}
\item [{algo\_executable}] Exactly what is specified in the \textbf{algo} argument in the scenario file.
 
\item [{instance\_name}] The name of the problem instance we are executing
against.

\item [{instance\_specific\_information}] An arbitrary string associated
with this instance as specified in the \textbf{instance\_file }. If
no information is present then a ``0'' is always passed here. 

\item [{cutoff\_time}] The amount of time in seconds that the target algorithm
is permitted to run. It is the responsibility of the callee
to ensure that this is obeyed. It is not necessary that that the actual
algorithm execution time (wall clock time) be below this value (\eg{If the algorithm needs to cleanup, or it's only possible to terminate the algorithm at certain stages}).

\item [{cutoff\_length}] A domain specific measure of when the algorithm should consider itself done.

\item [{seed}] A positive integer that the algorithm should use to seed
itself (for reproducibility). ``-1'' is used when the algorithm is \textbf{deterministic}

\item [{param}] A setting of an active parameter for the selected configuration
as specified in the Algorithm Parameter File. SMAC will only pass
parameters that are active. Additionally SMAC is not guaranteed
to pass the parameters in any particular order. The exact format for
each parameter is:\\
\texttt{-name~value}\footnote{The target algorithm will see the value as a single argument, even if it contains spaces or should otherwise be treated as more than one argument, i.e. to execute this in a shell you would see -name 'value', to ensure that the value is passed as a single argument. Older versions also passed the single quote}

\end{description}

All of the arguments above will always be passed, even if they are inapplicable, in which case a dummy value will be passed. \\
\subsubsection*{Environment Variables}
\label{sec:exec-env}
Recent versions of SMAC also set the following environment variables, which shouldn't be considered input to the solver but  relate to the execution in some way. When implementing your wrapper you can entirely disregard this section unless you need some advanced features.

\begin{center}
\begin{tabular}{| c | p{10cm} |}
\hline
\textbf{Environment Variable} & \textbf{Purpose} \\
\hline
\textbf{ AEATK\_CONCURRENT\_TASK\_ID } &  A zero indexed value of which concurrent run SMAC is executing. No two concurrent runs will see the same value, but subsequent runs will see the same value. This is mainly intended to allow the wrapper to manage CPU affinities. \\
\hline
\textbf{ AEATK\_SET\_TASK\_AFFINITY } & This environment variable is NOT set by SMAC, or used by SMAC but is used internally by some wrappers to ensure that AEATK\_CONCURRENT\_TASK\_ID is read and the task is tied to a specific core. It is strongly recommended that you set this value to ``1'' which your wrapper then reads to know to set the affinity properly. This isn't enabled by default, because some clusters, notably SGE do not set job affinities properly and so parallel jobs will get tied to the same core.\\
\hline 
\textbf{ AEATK\_PORT} & The wrapper can send in progress up dates to SMAC to the localhost via UDP on this port. The message format is a single double value. While SMAC doesn't directly use this presently, other utilities such as \texttt{algo-test} do, and future versions may, and some advanced options may preform better with this set. \\
\hline 
\textbf{ AEATK\_CPU\_TIME\_FREQUENCY } & Signifies how often updates to the runtime should be sent. This is only a hint, and roughly should be treated as: there is no point in sending updates more frequently than this value in seconds. \\
\hline
\textbf{ AEATK\_EXECUTION\_UUID } & A UUID associated with the particular invocation of the wrapper. The primary purpose of this is facilitate identify every process associated with a specific invocation of a wrapper on a particular computer. This is primarily used in conjunction with the \textbf{-$~\!\!$-cli-kill-by-environment-cmd} option. If this environment variable already exists, then another one will be chosen, so do not rely on this particular variable being set. You should ensure that environment variables are passed to all sub processes however, so that they can be killed accordingly.\\
\hline
\end{tabular}
\end{center}



%%%%%%%%%%%%%%%%%%%%%%%%%%%%%%%%%%%%%%%%%%%%%%%%%%%%%%%%%%%%%%%%%%%%
\subsubsection{Output}\label{sec:wrapper_output}
%%%%%%%%%%%%%%%%%%%%%%%%%%%%%%%%%%%%%%%%%%%%%%%%%%%%%%%%%%%%%%%%%%%%

The Target Algorithm is free to output anything, which will be ignored
but must at some point output a line (only once) in the following
format\footnote{Other strings are also permissible, but will one day be replaced. Most notably ``Result for ParamILS:''}:%



%ParamILS in not a typo. While other values are possible including
%SMAC, HAL. ParamILS is probably the most portable. The exact Regex
%that is used in this version is: \textbf{\^\textbackslash{}s{*}(Final)?%\textbackslash{}s{*}{[}Rr{]}esult\textbackslash{}s+(?:([Ff]or)|([oO]f))\textbackslash{}s+(?:(HAL)|(ParamILS)|(SMAC)|([Tt]his
%[wW]rapper))}}:%

\texttt{Result of this algorithm run: <status>, <runtime>, <runlength>, <quality>, <seed>, <additional rundata>} 
\begin{description}

\item [{status}] Must be one of \texttt{SAT} (signifying a successful run that was satisfiable), \texttt{UNSAT} (signifying a successful run that was unsatisfiable), \texttt{TIMEOUT} if the algorithm didn't finish within the allotted time, \texttt{CRASHED} if something untoward happened during the algorithm run, or \texttt{ABORT} if something prevents the target algorithm for successfully executing and it is believed that further attempts would be futile. 

SMAC does not differentiate between \texttt{SAT} and \texttt{UNSAT} responses, and the primary use of these is historical and serves as a check that the algorithm is executing correctly by outputting whether the instance in question is satisfiable or not. See the \textbf{-$\!~$-verify-sat} option for information on how to utilize this feature. 
\\
\textsc{Note:} SMAC by default crashes if the wrapper ever reports SAT and UNSAT for the same instance across runs. Occasionally edge cases in exposed parameters are tripped and turn a solver buggy, and so this safe guard exists to help detect if this is occurring. To change this behaviour use the \textbf{-$\!~$-check-sat-consistency} and \textbf{-$\!~$-check-sat-consistency-exception} options.

SMAC also supports reporting \texttt{SATISFIABLE} and \texttt{SUCCESS} for \texttt{SAT} and \texttt{UNSATISFIABLE} for \texttt{UNSAT}.
\\ \textsc{Note:} These are only aliases and SMAC will not preserve which alias was used in the log or state files.

\texttt{ABORT} can be useful in
cases where the target algorithm cannot find required files, or a
permission problem prevents access to them. This will cause SMAC to
stop running immediately. Use this option with care, it should only be reported when the algorithm knows for \textsc{certain} that subsequent results may fail. For things like sporadic network failures, and other cosmic-ray induced failures, one should consider using \texttt{CRASHED} in combination with the \textbf{-$\!~$-retry-crashed-count}  and \textbf{-$~\!\!$-abort-on-crash} options, to mitigate these.

In other files or the log you may see the following following additional types used. \texttt{RUNNING} which signifies a result that is currently in the middle of being run, and \texttt{KILLED} which signifies that SMAC internally decided to terminate the run before it finished. These are internal values only, and wrappers are NOT permitted to output these values. If these values are reported by the wrapper, it will be treated as if the run had status \texttt{CRASHED}.

\item [{runtime}] The amount of CPU time used during this algorithm run.
SMAC does not measure the CPU time directly, and this is the amount
that is used with respect to \textbf{tunerTimeout}. You may get
unexpected performance degradation when this amount is heavily under
reported \footnote{This typically happens when targeting very short algorithm
runs with large overheads that aren't accounted for.}. 

\textsc{Note:}The \textbf{runtime }should always be strictly less
than the requested \textbf{cutoff\_time } when reporting \texttt{SAT}or \texttt{UNSAT}. The runtime must be strictly greater than zero (and not NaN).

If an algorithm reports \texttt{TIMEOUT} or \texttt{CRASHED} the algorithm can report the actual CPU time used, and SMAC will treat it correctly as a timeout for optimization purposes, but count the actual time for \textbf{-$\!~$-tunertime-limit} purposes.

\item [{runlength}] A domain specific measure of how far the algorithm
progressed. This value must be from the set: ${-1} \cup [0,+\infty]$.

\item [{quality}] A domain specific measure of the quality of the solution. This value needs to be 
from the set: $(-\infty, +\infty)$. 

\textsc{Note:} Keep in mind that SMAC will attempt to \emph{minimize} this value. If you would like to maximize this value, your wrapper should subtract your quantity from some constant known to be larger than any value.

\textsc{Note}: In certain cases, such as when using log transforms in the model, this value must be: $(0, +\infty)$.

\item [{seed}] The seed value that was used in this target algorithm execution.


\textsc{Note:} This seed argument is ignored by SMAC as of version 2.06.02. SMAC will always use the requested seed internally and so you can ignore this output.  Note previous versions, as well as ParamILS and other applications may still require that this matches.

\item[{additional rundata}] A string (not containing commas, or newline characters) that will be associated with the run as far as SMAC is concerned. This string will be saved in run and results file (Section \ref{subsec:state-files}). \\
\textsc{Note}:\textbf{additional rundata} is not compatible with ParamILS at time of writing, and so wrappers should not include this or the preceding comma if they wish to be compatible.


\end{description}
All fields except for \textbf{additional rundata} are mandatory. If the field is not applicable for your scenario a 0 can be substituted.

%%%%%%%%%%%%%%%%%%%%%%%%%%%%%%%%%%%%%%%%%%%%%%%%%%%%%%%%%%%%%%%%%%%%
\subsection{Wrapper Output Semantics}
%%%%%%%%%%%%%%%%%%%%%%%%%%%%%%%%%%%%%%%%%%%%%%%%%%%%%%%%%%%%%%%%%%%%

As SMAC is entirely insulated from the target algorithm execution by the wrapper it is up to the wrapper to 
ensure that constraints with respect to the cutoff and runlength are enforced. Occasionally wrappers may not properly enforce these constraints and SMAC will need to somehow handle these cases. The following table outlines how SMAC transforms these values and details what value is used in various parts of SMAC. In future versions some parts of this table may in fact change, and so it is best to ensure that your wrapper is well behaved.\\

 \textsc{Note}: The cutoff time in the table is the amount of time SMAC schedules the run for, the scenario cutoff time is denoted as $\kappa_{max}$.

\begin{center}
\begin{tabular}{|l | l | l  ||  c | c | c |}
\hline
\textbf{status} & \textbf{cutoff} ($\kappa$) & \textbf{runtime} ($r$) & Tuner Time & PAR10 Score & Model \\
\hline
\hline
* & *  & $(-\infty, 0)$ & \multicolumn{3}{c|}{\textbf{EXCEPTION THROWN}} \\ %%Enforced by AbstractAlgorithmRun
\hline
\texttt{ABORT} &  *  & $[0, \infty)$  & \multicolumn{3}{c|}{\textbf{EXCEPTION THROWN}} \\ %%Enforced by AbstractAlgorithmRun
\hline
\texttt{CRASHED} & *  & $[0, \infty)$ & $r$ & $10\cdot\kappa_{max}$ & $10\cdot\kappa_{max}$ \\ %%Response value turns into 10*\kappa
\hline
{\texttt{SAT}, \texttt{UNSAT}} & $\kappa\leq\kappa_{max}$  & $[0,0.1]$ & 0.1 & $r$ & $r$ \\ %%Tuner Time limit enforced by CPULimitCondition
\hline
{\texttt{SAT}, \texttt{UNSAT}} & $\kappa\leq\kappa_{max}$ & $[0.1,\kappa)$ & $r$ & $r$ & $r$ \\ %%Normal case
\hline
{\texttt{SAT}, \texttt{UNSAT}} & $\kappa\leq\kappa_{max}$ & $[\kappa, \kappa_{max})$ & $r$ & $r$ & $r$ \\ %%PAR Score is done by the OverallObjective, and model is in SMAC.java
\hline
{\texttt{SAT}, \texttt{UNSAT}} & $\kappa\leq\kappa_{max}$ & $[\kappa_{max},\infty)$ & $r$  & $10\cdot\kappa_{max}$ & $10\cdot\kappa_{max}$ \\ %%PAR Score is done by the OverallObjective, and model is in SMAC.java

\hline
{\texttt{TIMEOUT}} & $\kappa<\kappa{max}$ & $[0,\infty)$ & $r$ & $\kappa$ & $\kappa$ \\ %%This seems suspect but the RunObjective always uses this the cutoff time indiscriminately.
\hline
{\texttt{TIMEOUT}} & $\kappa=\kappa{max}$ & $[0,\infty)$ & $r$  & $10\cdot\kappa_{max}$ & $10\cdot\kappa_{max}$\\
\hline
\end{tabular}
\end{center}

A description of the locations is as follows:

\begin{description}
\item[Tuner Time] The amount of time that will be subtracted from the remaining tuner time limit given in the scenario.
\item[PAR10 Score] The value that will be used for empirical comparisons between configurations.
\item[Model] 	The value that will be used to build the model.
\end{description}




%%%%%%%%%%%%%%%%%%%%%%%%%%%%%%%%%%%%%%%%%%%%%%%%%%%%%%%%%%%%%%%%%%%%
\subsection{Wrappers \& Native Libraries}
%%%%%%%%%%%%%%%%%%%%%%%%%%%%%%%%%%%%%%%%%%%%%%%%%%%%%%%%%%%%%%%%%%%%
\label{sec:exec-options}
In order to optimize an algorithm, SMAC needs a method of invoking it. While modifying the code
to manage the timing and input mechanisms manually is possible, this can sometimes be invasive and difficult to manage. There exist three other methods that one could consider using.

\begin{description}
\item [{Wrappers}] Executable Scripts that manage the resource limits automatically
and format the specified string into something usable by the actual
target algorithm. This approach is probably the most common, but among
its drawbacks are the fact that they often rely on third party scripting languages,
and for smaller execution times have a large amount of overhead that
may not be accounted for as far as the \textbf{tunerTimeout} limit is concerned. Most of the examples included in SMAC use this approach, and the wrappers included can be adapted for your own
projects.

\textsc{Note:} When writing wrappers it is important not to poll the output stream of the target algorithm, especially if there is lots of output. Doing so often results in lock-contention and significantly modifies the runtime performance of the algorithm enough that the resulting configuration is not well tuned to the real algorithm's performance.

%\item [{Native~Libraries~Augmentation}] Libraries exist (See: \textbf{TBD}) for C and Java currently that facilitate adding the required functionality directly to the code. While
%parsing the arguments into the necessary data structures is still required, they do manage the timing and output requirements in most cases. Unlike the previous approach however, certain crashes may not allow the the values to be outputted (\eg{ a segmentation fault occurs}).

\item [Inter~Process~Communication] Introduced in SMAC v2.06.02, the \textbf{IPC} TAE can be selected via the \textbf{-$~\!$-tae} option. It will essentially use various forms of interprocess communication to notify some other process about the run to do, and will wait for it to respond. Presently the only existing mechanism is over UDP, but other methods are possible. For example:

\begin{verbatim}
./smac --scenario-file <file> --seed 1 --tae IPC 
--ipc-mechanism UDP --ipc-remote-port 5050 
--ipc-remote-host localhost
\end{verbatim}

Will use sent the required information to localhost on port 5050. It will then wait for a response, and parse that response as a string. For more details see the section "Inter-Process Communication Target Algorithm Evaluator Options", in the Appendix.


\item [{Target~Algorithm~Evaluators}]	This is probably the most powerful, but also the most complicated approach. SMAC is architected in a way that makes it fairly simple to replace the mechanism for execution with something completely custom. This can be done without even recompiling SMAC by creating a new implementation of the \texttt{TargetAlgorithmEvalutor} interface, which is responsible for converting run requests (\texttt{RunConfig} objects) into run results (\texttt{AlgorithmRun} objects). Both the input and output objects are simple \emph{Value Objects} so the coupling between SMAC and the rest of your code is almost zero with this approach.  For more information see \ref{sec:target-algorithm-evaluators}


\end{description}
\end{document}