\documentclass[manual.tex]{subfiles}\begin{document}
\subsection{Running SMAC}
To get started with an existing configuration scenario you simply
need to execute smac as follows:

\begin{verbatim}
./smac --scenario-file <file> --seed 1
\end{verbatim}

This will execute SMAC with the default options on the scenario specified in the file. 
Some commonly-used non-default options of SMAC are described in this section. The \textbf{-~$\!$-seed} argument controls the seed and names of output files (to support parallel independent runs). The \textbf{-~$\!$-seed-offset} argument lets you keep the output folders names simple while varying the actual seed of SMAC. The seed argument is also optional and will automatically be chosen if not set.

\subsection{Testing the Wrapper} \label{sec:test}
SMAC includes a method of Testing Algorithm Execution, via the \texttt{algotest} utility. It takes the required scenario options \footnote{Unfortunately it cannot read scenario files currently}

For example:
\begin{verbatim} 
./algotest --scenario-file <scenario> --instance <instance>  
--config <config string> -P[name]=[value] -P[name]=[value]...
\end{verbatim}

Some parameters deserve special mention:
\begin{enumerate}
\item The config string syntax is a single string with ``-name=`value' '' ... you can also specify \texttt{RANDOM} which will generate a random configuration or \texttt{DEFAULT} which will generate  the default configuration.

\item The \texttt{-P} parameters are optional and allow overriding specific values in the configuration (this is useful primarily for RANDOM and DEFAULT, to allow you to set certain values). To set the \texttt{sort\-algo} parameter to \texttt{merge} you would specify \texttt{\-Psort\-algo=merge}.
\end{enumerate}

\subsection{Verifying the Scenario} \label{sec:verify}
SMAC includes a utility that allows you to test the scenario. It is currently \texttt{BETA} but does a bit more sanity checks than SMAC will normally do.

For example:
\begin{verbatim}
./verify-scenario --scenarios ./scenarios/*.txt --verify-instances true
\end{verbatim}

The utility has some limitations however:
\begin{enumerate}
\item 	It currently does not check test instances
\item	Scenario files can specify non-scenario options in SMAC (and some of the example scenarios in fact do), this utility is not aware of them, and will report an error.
\end{enumerate}



%\subsection{ROAR Mode}

%\begin{verbatim}
%./smac --scenarioFile <file> --exec-mode ROAR --seed 1
%\end{verbatim}

%This will execute the ROAR algorithm, a special case of SMAC that uses an empty model and random selection of 
%configurations. See \cite{HutHooLey11-SMAC} for details on ROAR.

%\subsection{Adaptive Capping}
%\begin{verbatim}
%./smac --scenarioFile <file> --adaptive-capping true --seed 1
%\end{verbatim}
%Adaptive Capping (originally introduced for ParamILS~\cite{ParamILS-JAIR}, but also applicable in SMAC~\cite{HutHooLey11-censoring}) will cause SMAC to only schedule algorithm runs for as long as is needed to determine whether they are better than the current incumbent. Without this option, each target algorithm runs up to the runtime specified in the configuration scenario file \textbf{-~$\!$-algo-cutoff-time}.

%\noindent{}\textsc{Note:} Adaptive Capping should only be used when the \textbf{-~$\!$-run-obj} is RUNTIME.
%Adaptive capping can drastically improve SMAC's performance for scenarios with a large difference between 
%\textbf{-~$\!$-algo-cutoff-time} and the runtime of the best-performing configurations.


\subsection{Wall-Clock Limit}\label{sec:wall-clock}
\begin{verbatim}
./smac --scenario-file <file> --wallclock-limit <seconds> --seed 1
\end{verbatim}
SMAC offers the option to terminate after using up a given amount of wall-clock time. This option is useful to limit the overheads of starting target algorithm runs, which are otherwise unaccounted for.
This option does not override \textbf{-~$\!$-tunertime-limit} or other options that limit the duration of the configuration run; whichever termination criterion is reached first triggers termination. 

\subsection{Change Initial Incumbent}\label{sec:initial-incumbent}
\begin{verbatim}
./smac --scenario-file <file>  --initial-incumbent <config string>
\end{verbatim}

SMAC offers the option to specify the initial incumbent, and by default uses the default configuration specified in the parameter file. The argument to \textbf{-~$\!$-initial-incumbent} follows the same conventions as in Section \ref{sec:test}.

\subsection{State Restoration}\label{sec:state-restoration}
\begin{verbatim}
./smac --scenario-file <file> --restore-scenario <dir> 

\end{verbatim}
SMAC will read the files in the specified directory and restore 
its state to that of the saved SMAC run at the specified iteration.
Provided the remaining options (e.g. \textbf{-~$\!$-seed}, \textbf{-~$\!$-overall\_obj}) are set identicially, SMAC should continue along the same trajectory.

This option can also be used to restore runs from SMAC v1.xx (although due to the lossy nature of Matlab files and differences in random calls you will not get the same resulting trajectory). By default the state can be restored to iterations that are powers of 2, as well as the 2 iterations prior to the original SMAC run stopping. 
If the original run crashed, additional information is saved, often allowing you to replay the crash.

\textsc{Note:} When you restore a SMAC state, you are in essence preloading a set of runs and then running the scenario. In certain cases, if the scenario has been changed in the meantime, this may result in undefined behaivor. Changing something like \textbf{-~$\!$-tunertime-limit} is usually a safe bet, however changing something central (such as \textbf{-~$\!$-run-obj}) would not be.

To check the available iterations that can be restored from a saved
directory, use:
\begin{verbatim}
./smac-possible-restores <dir>
\end{verbatim}


\subsection{Warm-Starting the Model}
\begin{verbatim}
./smac --scenario-file <file> --warmstart <foldername>
\end{verbatim}
Using the same state data as in Section \ref{sec:state-restoration}, you can also just choose to warm-up the model with previous runs. Instead of the \textbf{-~$\!$-restore-scenario} option use \textbf{-~$\!$-warmstart} instead. SMAC will operate normally, but when building the model the above data will also be used. Please keep in mind the following.\\

\textsc{Note}: If the execution mode is ROAR, this option has no effect.\\

\textsc{Warning}: Due to design limitations of the state restoration format in this version of SMAC you cannot / should not have any differences between the instance distribution used to warmstart the model, and the instance distribution we are configuring against. In the best case you will simply get a random exception at some point (perhaps a \texttt{NullPointerException}), and in the worst case it will just load the model with junk. \\

\textsc{Tip}: The included state-merge utility allows you to easily merge a bunch of different runs of SMAC into one state that you can use for a warm start.

\subsection{Named Rungroups}
\begin{verbatim}
./smac --scenario-file <file> --rungroup <foldername> 
\end{verbatim}
All output is written to the folder \texttt{$<$foldername$>$}; runs differing in \textbf{-~$\!$-seed} will yield different output files in that folder.

\subsection{More Options}
By default SMAC only displays BASIC usage options, other options are INTERMEDIATE, ADVANCED, and DEVELOPER. Be warned that there are a bunch of options and some of the more advanced and developer options may cause SMAC to perform very poorly.

\begin{verbatim}
./smac --help-level INTERMEDIATE
\end{verbatim}


\subsection{Shared Model Mode}
\label{subsec:sharedModelMode}
\textcolor{red}{\textbf{NOTE:}} \textbf{Please read this full section before deciding to use this option.}
\vspace{5pt}

SMAC has an advanced option that essentially allows multiple runs of SMAC to share data and construct better models quickly. 

\begin{verbatim}
./smac --scenario-file <file> --shared-model-mode true
\end{verbatim}

There are a couple of things to keep in mind when running with this option:

\begin{enumerate}

\item The first is that the different SMAC runs need to be using the exact same scenario, that MUST have different seeds. Even small inconsequential differences may cause this to fail (for instance if the location on different machines and the path to execute the target algorithm is different). SMAC should in most cases recover gracefully and just ignore the incompatible run data, but it is possible that the SMAC run may be corrupted. 

\item The shared file system between clusters needs to allow a file that is being written on one machine to be read on another machine. We've had reports that on some file systems (AFS) with some locking policies you cannot do this. In this case you can still benefit from this mode but it might need to be a little more coarse. After doing a first set of runs, you can then do a second batch with the --share-run-data, they won't be able to read the second batches data, but they should be able to read the first batches.

\item The frequency with which runs are re-read is controlled via the \textbf{-~$\!$-shared-model-mode-frequency} which defaults to 5 minutes, depending on your scenario and the amount of data you need you may want to set it lower. If your runs take considerably longer than 5 minutes there is no need to increase this, the data is only re-read at most once for every local algorithm run, and the above is designed to prevent hitting the file system too frequently

\item You probably will need to increase the amount of memory you give to SMAC, using the \texttt{SMAC\_MEMORY} environment variable.

\item This mode turns N independent SMAC runs into N dependent runs of SMAC. This mode is designed to help get better performing configurations, it be inappropriate to treat these runs as independent samples from the same distribution for experimental purposes. What may be appropriate is to compare the experimental protocols, selecting the best performing run of independent SMAC on the training set, versus the best performing run of dependent SMAC on the training set, and reporting their values on the testing set.

\item This mode does not require that different runs of SMAC execute concurrently at all. At one other extreme case runs could happen sequentially, this mode would just be an easier way of running SMAC, flushing the run data, but warm starting the model with the previous. At the other  extreme case, and the case we have some preliminary experiments for, all the runs are started at the exact same time. In this case, there was a substantial boost in median performance (but see previous point why this is misleading), but also selecting the run with the best training instance over time generally resulted in as good or better performance. Another possibility is to have some runs not use this mode and have other runs with this enabled. Depending on the scenario, this might allow the models to maintain more diversity. Unfortunately the benefits and best practices of this option is unexplored at this time.
 	
 	 One advantage of this approach is that if you only care about getting a good configuration quickly (without worrying about reproducibility), you can schedule the runs of SMAC to the cluster independently, which should make them quicker to dispatch and yet still benefit from the shared data.
 
\end{enumerate}
When in this mode you should see messages like the following which indicate a new source of runs was detected:

\begin{verbatim}
[INFO ] Detected new sources of shared runs : [live-rundata-1.json, live-rundata-2.json]
\end{verbatim}

This indicates that we have started to read runs from the above files.


At the end of a run you will see a line like:

\begin{verbatim}
[INFO ] At shutdown: ./smac-output/branin-scenario/live-rundata-3.json 
had 10 runs added to it
[INFO ] At shutdown: we retrieved atleast 20 runs and added them to
 our current data set [live-rundata-1.json=>10, live-rundata-2.json=>10]
\end{verbatim}

\subsection{Offline Validation}

SMAC includes a tool for the offline assessment of incumbents selected during the configuration process.
By default, given a test instance file with $N$ instances, SMAC performs $\approx$ 1\,000 target algorithm validation runs per configuration (rounded up to the nearest multiple of N).

By default, SMAC limits the number of seeds used in validation runs to 1\,000 seeds per instance. This number can be changed as in the following example:
\begin{verbatim}
./smac --scenario-file <file> --num-seeds-per-test-instance 50
\end{verbatim}
(This parameter does not have any effect in the case of instance/seed files.)


\subsubsection{Limiting the Number of Instances Used in a Validation Run}

To use only some of the instances or instance seeds specified you can limit them with the \textbf{-~$\!$-num-test-instances} parameter. When this parameter is specified, SMAC will only use the specified number of lines from the top of the file, and will keep repeating them until enough seeds are used: 
\begin{verbatim}
./smac --scenario-file <file> --num-test-instances 10
\end{verbatim}
For instance files containing seeds, this option will only use the specified number of instance seeds in the file.

\subsubsection{Disabling Validation}
Validation can be skipped alltogether as follows:
\begin{verbatim}
./smac --scenario-file <file> --seed 1 --validation false
\end{verbatim}

\subsubsection{Standalone Validation}
SMAC also includes a method of validating configurations outside of a smac run.
You can supply a configuration using the \textbf{-~$\!$-configuration} option. All scenario options are applicable to the standalone validator, but check the usage screen to see all the options available \textsc{NOTE:} Some options while present are not applicable for validation but are presented anyway.

Here is an example call:
\begin{verbatim}
./smac-validate --scenario-file <file> --num-validation-runs 100
     --configuration <config string> --cli-cores 8 --seed 1
\end{verbatim}
%
Usage notes for the offline validation tool:
\begin{enumerate}
\item This validates against the test set only; the training instance set is not used.
\item By default this outputs into the current directory; you can change the output directory with the option \textbf{-~$\!$-rungroup}.
\item You can also validate against a trajectory file issued by \textbf{-$~\!$-trajectory-file} option. 


\end{enumerate}




\end{document}